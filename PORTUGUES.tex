\selectlanguage{brazilian}

%\let\footnote=\endnote

\section*{Pequenos poemas em prosa\break (o spleen de Paris)\protect\footnote{\uppercase{O} texto francês segue o estabelecido por \uppercase{M}arcel \uppercase{A}. \uppercase{R}uff para as \emph{\uppercase{Œ}uvres complètes de \uppercase{B}audelaire} editadas pelas \uppercase{E}ditions du \uppercase{S}euil, coleção \uppercase{L'I}ntégrale, \uppercase{P}aris, 1968. \uppercase{F}oram consultadas também a edição crítica dos \emph{\uppercase{P}etits poèmes en prose}, \uppercase{C}lassiques \uppercase{G}arnier, \uppercase{P}aris, 1980 (da qual foram seguidos os textos dos poemas \textsc{xliv} a \textsc{xlvii}) e \uppercase{N}ouveaux \uppercase{C}lassiques \uppercase{L}arousse, \uppercase{P}aris, 1971.}}

A Arsène Houssaye 

\hyphenation{fo-lhean-do}
\hyphenation{de-li-cio-sas}
\hyphenation{real-men-te}
\hyphenation{bea-ti-tu-de}
%\hyphenation{ex-pe-riên-cia}
\hyphenation{in-con-ve-nien-tes}
\hyphenation{mis-te-rio-sas}
\hyphenation{si-len-cio-sa-men-te}
\hyphenation{sua-ve-men-te}
%\hyphenation{crian-ça}
\hyphenation{re-crea-va}
\hyphenation{poe-ta}
\hyphenation{poe-ma}
\hyphenation{poe-sia}
\hyphenation{tem-pes-tuoso}
\hyphenation{pias-tra}
%\hyphenation{si-tua-ção}
\hyphenation{Fan-ciou-lle}
\hyphenation{pro-prie-da-de}
\hyphenation{es-ver-dea-do}
\hyphenation{pia-no}
\hyphenation{in-quie-tan-te}
\hyphenation{preen-chia}
\hyphenation{cria-tu-ra}
\hyphenation{sua-ve-men-te}


Meu caro amigo, estou lhe remetendo um pequeno trabalho do qual não se poderia dizer sem
injustiça que não tem pé nem cabeça, já que, pelo contrário, tudo nele
é ao mesmo tempo cabeça e pé, alternados e reciprocamente. Considere, eu
lhe peço, que admiráveis facilidades esta combinação oferece a todos
nós, a você, a mim e ao leitor. Podemos interromper onde quisermos, eu
meu devaneio, você o manuscrito, o leitor sua leitura; pois não
mantenho a recalcitrante vontade deste último suspensa ao fio
interminável de uma intriga supérflua. Retire uma vértebra, e os dois
pedaços desta tortuosa fantasia irão se juntar sem dificuldade.
Lacere"-a em diversos fragmentos, e verá que cada um deles pode
existir à parte. Na esperança de que algumas destas postas tenham vida
suficiente para agradá"-lo e diverti"-lo, ouso dedicar"-lhe a
serpente inteira.

Tenho uma pequena confissão a lhe fazer. Foi ao folhear, pela vigésima
vez no mínimo, o famoso \textit{Gaspard de La Nuit}, de Aloysius Bertrand (um
livro conhecido por você, por mim e alguns dos nossos amigos não
tem todo o direito de ser chamado famoso?), que me veio a ideia de
tentar algo análogo e aplicar à descrição da vida moderna, ou
melhor, de uma vida moderna e mais abstrata, o procedimento que ele
aplicou à pintura da vida antiga, tão estranhamente pitoresca.

Quem dentre nós não sonhou, nos seus dias de ambição, com o milagre de
uma prosa poética, musical sem rima nem ritmo, flexível e
desencontrada o bastante para se adaptar aos movimentos líricos da
alma, às ondulações do devaneio, aos sobressaltos da consciência?

É sobretudo da frequentação
das cidades imensas, do cruzamento de suas
inumeráveis relações que nasce este ideal obcecante. Você mesmo, 
caro amigo, não tentou traduzir numa canção o grito estridente do
vidraceiro, e expressar numa prosa lírica todas as aflitivas sugestões
que este grito manda para as águas"-furtadas, através das mais altas
brumas da rua?

Mas, para dizer a verdade, receio que minha inveja não me tenha trazido
sorte. Tão logo iniciei o trabalho, percebi que não só eu
permanecia bem distante do meu misterioso e brilhante modelo, como
também estava fazendo algo (se é que isto pode ser chamado de algo)
singularmente distinto, acidente de que qualquer outro além de mim
decerto se orgulharia, mas que só pode humilhar profundamente um
espírito que encara a maior honra do poeta como sendo o exato
cumprimento daquilo que ele projetou fazer.

\begin{flushright}
Afeiçoadamente seu,

C. B.
\end{flushright}

\setcounter{secnumdepth}{2} 

\quebra\section[O estrangeiro]{o estrangeiro}

--- A quem você ama mais, homem enigmático, me diga: seu pai, sua mãe, sua
irmã ou seu irmão?

--- Não tenho pai, nem mãe, nem irmã, nem irmão.

--- Seus amigos?

--- Está usando uma palavra cujo sentido para mim permanece
até hoje desconhecido.

--- Sua pátria?

--- Ignoro sob que latitude está situada.

--- A beleza?

--- Eu a amaria com prazer, deusa e imortal.

--- O ouro?

--- Eu o odeio como o senhor odeia a Deus.

--- Ei! O que é então que você ama, extraordinário estrangeiro?

--- Amo as nuvens\ldots\  as nuvens que passam\ldots\  lá, lá, adiante\ldots\  as
maravilhosas nuvens!

\quebra\section[O desespero da velha]{o desespero da velha}

A velhinha enrugadinha se sentiu toda faceira ao ver a linda criança a
quem todos faziam festa, a quem todo o mundo queria agradar; aquele lindo
ser tão frágil, como ela, a velhinha e, também como ela, sem dentes e
sem cabelo.

E aproximou"-se, querendo fazer gracinhas e trejeitos
simpáticos.

Mas a criança apavorada se debatia sob as carícias da boa mulher
decrépita, e enchia a casa com seus ganidos.

Então a boa velha recolheu"-se à sua solidão eterna, e chorava a um
canto, pensando: “Ah! para nós, infelizes velhas fêmeas,
passou a idade de agradar, mesmo aos inocentes; e horrorizamos as
criancinhas que queremos amar!”

\quebra\section[O “confiteor'']{o “confiteor''\protect\footnote{\uppercase{C}onfiteor, eu confesso, é o primeiro
termo da oração cristã de confissão.} do artista}

Que penetrantes sãos os finais de dia no outono! Ah! penetrantes até a
dor! Pois certas sensações deliciosas há das quais o indefinido não
exclui a intensidade; e ponta mais aguçada não há do que aquela do
Infinito.

Grande delícia esta de mergulhar o olhar na imensidão do céu e do mar!
Solidão, silêncio, incomparável castidade do azul! Uma pequena vela
estremecendo no horizonte e que por sua pequenez e seu isolamento
imita minha irremediável existência, melodia monótona do marulho, todas
estas coisas pensam por mim, ou eu penso por elas (pois na grandeza do
devaneio, o \textit{eu} se perde depressa!); elas pensam, digo, mas musical e
pitorescamente, sem argúcias, sem silogismos, sem deduções.

Estes pensamentos, porém, quer surjam de mim, quer jorrem das
coisas, em seguida se tornam demasiado intensos. A energia na volúpia cria
um mal"-estar e um sofrimento positivo. Meus nervos demasiado tensos
já não produzem mais que vibrações doloridas e estridentes.

E agora me consterna a profundeza do céu; sua limpidez me exaspera. A
insensibilidade do mar, a imutabilidade do quadro me revoltam\ldots\ 
Ah! Será preciso penar eternamente, ou o belo eternamente evitar?
Natureza, feiticeira sem dó, rival sempre vitoriosa, me deixe!
Pare de tentar meu orgulho e meus desejos! O estudo do belo é um duelo
em que o artista grita de pavor antes de ser vencido.

\quebra\section[Um engraçadinho]{um engraçadinho}

Era a explosão do ano novo: caos de lama e de neve, sulcado por mil
carruagens, resplandecendo de brinquedos e balas, formigando de
cupidezes e desesperos, delírio oficial de uma cidade grande, feito
para perturbar o cérebro do mais forte solitário.

Em meio àquele burburinho e alarido, trotava um burrico com
vivacidade, atormentado por um grosseirão armado de um chicote.

O burrico ia dobrando uma esquina quando um belo senhor enluvado,
lustrado, cruelmente engravatado e aprisionado em roupas novas em folha
inclinou-se cerimoniosamente perante o humilde animal e lhe disse,
tirando o chapéu: “Desejo"-lhe um bom e feliz ano-novo!”, 
voltando"-se depois para não sei que companheiros
com um ar de fatuidade, como a pedir"-lhes que unissem sua aprovação
ao seu contentamento.

O burrico não viu o belo engraçadinho e continuou a correr com zelo para
onde o chamava seu dever.

Quanto a mim, fui subitamente tomado de uma incomensurável raiva pelo
magnífico imbecil, que me pareceu concentrar em si todo o espírito da
França.

\quebra\section[O quarto duplo]{o quarto duplo}

Um quarto que semelha um devaneio, um quarto realmente \textit{espiritual}, cuja
atmosfera estagnante é levemente colorida de azul e cor"-de"-rosa.

A alma nele toma um banho de preguiça, aromatizado de pena e
desejo. --- É algo crepuscular, azulado e rosado; um sonho de volúpia durante um
eclipse.

Os móveis têm formas alongadas, prostradas, enlanguescidas. Os
móveis parecem sonhar; parecem dotados de vida
sonambúlica, como o vegetal e o mineral. Os tecidos falam uma língua
muda, como as flores, os céus, os sóis poentes.

Nas paredes, nenhuma abominação artística. Se comparada ao sonho puro, à
impressão não analisada, a arte definida, a arte positiva é uma
blasfêmia. Aqui, tudo tem a suficiente claridade e a deliciosa
obscuridade da harmonia.

Uma infinitésima fragrância da gama mais preciosa, a que se vem juntar 
muito leve umidade, vai nadando nesta atmosfera onde o espírito
sonolento é embalado por sensações de estufa quente.

A musselina chove em abundância frente às janelas e à cama, se expande em cascatas nevadas. Nesta cama, está deitada a
Ídola,\protect\footnote{ Sendo \textit{idole} em francês uma palavra feminina, preferiu"-se manter o
gênero em português, em respeito ao contexto e a despeito da estranheza
que possa causar.} a soberana dos sonhos. Mas como ela veio parar
aqui? Quem a trouxe? Que poder mágico a instalou neste trono de
volúpia e devaneios? Que importa? Ela está aqui! Eu a reconheço.

\quebra

São mesmo seus olhos, cuja chama atravessa o crepúsculo; sutis
e terríveis olhinhos, que reconheço por sua assustadora malícia! Atraem, subjugam, devoram o olhar do imprudente que os contempla. Com
frequência estudei essas estrelas negras que clamam por curiosidade
e admiração.

A que demônio benigno devo o estar assim rodeado de mistério, 
silêncio, paz e perfumes? Oh, beatitude! Isso que geralmente
chamamos de vida nada tem em comum, mesmo em sua expansão mais venturosa,
com esta vida suprema da qual tenho ciência agora, e que saboreio
minuto a minuto, segundo a segundo!

Não! Já não existem minutos! Já não existem segundos! O tempo
desapareceu; quem reina é a Eternidade, uma eternidade de delícias!

Mas uma batida terrível, pesada, ressoou na porta e, qual nos sonhos
infernais, tive a impressão de levar uma enxadada no estômago.

E então, um Espectro entrou. Um bedel vindo me torturar em nome da
lei, uma infame concubina vindo clamar miséria e somar as
trivialidades da sua vida às dores da minha; ou então o estafeta de um
diretor de jornal cobrando a sequência do manuscrito.

O quarto paradisíaco, a ídola, a soberana dos sonhos, a
Sílfide, como dizia o grande
René,\protect\footnote{ François"-René de Chateaubriand (1768"-1848), \textit{Mémoires
doutre"-tombe}, \textsc{i}, \textsc{iii}, 11.}
esta magia toda desapareceu com a batida
brutal do Espectro.

Horror! Estou lembrando! Sim! Estou! Este casebre, morada do eterno
tédio, é realmente o meu. Eis os móveis

\noindent{}tolos, poeirentos, truncados; a
lareira sem chama e sem brasa, manchada de cuspe; as tristes janelas em
que a chuva traçou sulcos na poeira; os manuscritos, rasurados ou
incompletos; a folhinha em que o lápis marcou as datas sinistras!

E este aroma de outro mundo, com que eu me embriagava com
sensibilidade aprimorada, ai! foi substituído por um fétido cheiro
de fumo misturado a não sei que nauseabundo bolor. Respira"-se agora
aqui o ranço da desolação.

Neste mundo estreito, mas tão cheio de desencanto, um único objeto conhecido
me sorri: a garrafinha de láudano, uma velha e terrível amiga e, ai!
como todas as amigas, fecunda em carícias e traições.

Oh! Sim! O Tempo reapareceu; o Tempo agora reina soberano; e com o
hediondo velho, voltou todo o seu demoníaco cortejo de Lembranças,
Desgostos, Espasmos, Medos, Angústias, Pesadelos, Raivas e Neuroses.

Garanto que os segundos agora são forte e solenemente acentuados, e
cada um deles diz, brotando do relógio: ``Eu sou a Vida, a
insuportável, a implacável Vida!''

Há na vida humana um único Segundo cuja missão é anunciar uma \textit{boa
nova}, a boa nova que causa em todos nós um medo inexplicável.

Sim! O Tempo reina; ele retomou sua brutal ditadura. E me empurra, como
se eu fosse um boi, com seu duplo agui-

\quebra

\noindent{}lhão. --- ``Ora!
Eia! burrico! Ora, sue, escravo! Ora, viva, danado!''

\quebra\section[Cada qual com sua quimera]{cada qual com sua quimera}

Sob um vasto céu cinzento, numa vasta planície poeirenta, sem
caminhos, sem gramados, sem uma urtiga, sem um cardo, deparei com vários
homens que andavam curvados.

Cada um deles carregava nas costas uma enorme
Quimera,\protect\footnote{  A Quimera, na mitologia grega, era um monstro que cuspia fogo, dotado
de cabeça e peito de leão, tronco de cabra e cauda de dragão.} pesada como um saco de farinha ou
carvão, ou como os apetrechos de um soldado da infantaria romana.

Mas a monstruosa besta não era um peso inerte; pelo contrário, envolvia
e oprimia o homem com seus músculos elásticos e possantes;
enganchava"-se com as duas vastas garras no peito de sua montaria;
sua cabeça fabulosa sobressaía acima da fronte do homem, como um
daqueles capacetes horríveis com que os guerreiros antigos
contavam acirrar o terror do inimigo.

Interroguei um desses homens, e perguntei"-lhe aonde iam assim.
Respondeu"-me que de nada sabia, nem ele nem os outros, mas que
evidentemente iam para algum lugar, já que eram impelidos por uma invencível
necessidade de andar.

Coisa curiosa de se notar: nenhum dos viajantes parecia irritado com a
besta feroz pendurada em seu pescoço e grudada em suas costas;
até parecia considerá-la como parte de si mesmo. Todos
aqueles rostos cansados e sérios não demonstravam nenhum desespero; sob a
cúpula \textit{spleenética} do céu, os pés mergulhados na poeira de um solo
tão desolado quanto este céu, caminhavam com o semblante
resignado de quem está condenado a ter sempre esperança.

E o cortejo passou ao meu lado e afundou na atmosfera do horizonte, no
lugar em que a superfície arredondada do planeta se esquiva à
curiosidade do olhar humano.

E durante alguns instantes, teimei em tentar compreender aquele mistério;
mas em seguida a irresistível Indiferença se abateu sobre mim, e me
deixou mais duramente oprimido do que eles próprios por suas
esmagadoras Quimeras.

\quebra\section[O louco e a vênus]{o louco e a vênus}

Que dia admirável! O vasto parque se pasma sob o olho ardente do sol,
como a juventude sob a dominação do Amor.

O êxtase universal das coisas não se expressa por ruído algum; as
próprias águas estão como adormecidas. Bem diferente das festas
humanas, esta é uma orgia silenciosa.

Dá a impressão que uma luz sempre crescente faz mais e mais
resplandecerem os objetos; que as flores excitadas ardem no desejo de
competir com o azul do céu pela energia de suas cores, e que o calor,
tornando os aromas visíveis, os faz subir como fumaça rumo ao astro.

Em meio a este gozo universal, porém, avistei um ser aflito.

Aos pés de uma Vênus colossal, um desses loucos artificiais, desses
bufões voluntários incumbidos do riso dos reis quando o Remorso ou o
Tédio os obceca, vestindo um traje vistoso e ridículo, cabeça
coberta de chifres e guizos, todo encolhido junto ao pedestal, ergue os
olhos cheios de lágrimas para a Deusa imortal.

E seus olhos dizem: ``Sou o último e o mais solitário dos
humanos, privado de amor e de amizade, e nisto bem inferior ao mais
imperfeito dos animais. No entanto, também eu fui criado para entender
e sentir a imortal Beleza! Ah! Deusa! Tende piedade da minha tristeza e
do meu delírio!''

Mas a implacável Vênus olha ao longe, para não sei o quê, com seus olhos
de mármore.

\quebra\section[O cão e o frasco]{o cão e o frasco}

``Meu bom cão, meu belo cão, meu querido cachorrinho,
aproxime"-se e venha cheirar um perfume excelente, comprado na melhor
perfumaria da cidade.''

E o cão, meneando a cauda, o que, acredito, é nestas pobres criaturas o
sinal correspondente ao riso e ao sorriso, se aproxima e traz,
curioso, seu úmido focinho ao frasco destampado; então, subitamente
recuando de terror, late para mim em forma de censura.

``Ah! cão miserável, se eu lhe tivesse oferecido um pacote
de excrementos, você o teria farejado com delícia, e talvez devorado.
Assim, mesmo você, companheiro indigno de minha triste vida, se parece
com o público, ao qual não se deve jamais apresentar delicados perfumes
que o exasperam, mas lixo cuidadosamente escolhido.''

\quebra\section[O mau vidraceiro]{o mau vidraceiro}

Existem naturezas puramente contemplativas e totalmente impróprias para
a ação que, no entanto, movidas por algum misterioso e desconhecido impulso,
agem às vezes com uma rapidez de que elas próprias se julgariam
incapazes.

Como quem, temendo encontrar com o zelador uma notícia aflitiva,
ronda por uma hora covardemente frente à porta de casa sem ousar
entrar, como quem guarda durante quinze dias uma carta sem
abri"-la, ou só ao fim de seis meses se conforma em efetuar um
empreendimento necessário desde um ano, elas se sentem às vezes
bruscamente precipitadas para a ação por uma força irresistível, qual a
flecha de um arco. O moralista e o médico, que afirmam saber de tudo,
não podem explicar de onde vem tão de súbito uma tão louca energia
nessas almas preguiçosas e voluptuosas, e como é que elas, incapazes de
cumprir as coisas mais simples e mais necessárias, encontram em dado
momento uma coragem de luxo para executar os atos mais absurdos e até,
muitas vezes, os mais perigosos.

Um dos meus amigos, o mais inofensivo sonhador que já existiu, ateou
fogo certa vez a uma floresta, para ver, dizia, se o fogo pegava tão facilmente como se costuma afirmar. Dez vezes consecutivas, a
experiência falhou; na décima primeira, porém, foi por demais bem
sucedida.

Outro poderá acender um charuto do lado de um barril de pólvora, \textit{para ver,
para saber, para tentar o destino}, para forçar a si mesmo a dar
provas de energia, para dar uma de jogador, para experimentar os prazeres da %dara\dar?
ansiedade, por nada, por capricho, por desocupação. 

É uma espécie de energia que jorra do tédio e do devaneio; e aqueles em quem se manifesta tão inopinadamente são em geral, como 
disse, os mais indolentes e sonhadores dos seres.

Outro, tímido a ponto de baixar os olhos mesmo diante do olhar dos
homens, de precisar juntar toda a sua pobre vontade
para entrar num bar ou passar em frente a uma bilheteria de teatro, onde
os fiscais lhe parecem investidos da majestade de Minos, Éaco e
Radamanto,\protect\footnote{  Minos, Éaco e Radamanto eram, na mitologia grega, os três juízes do
Inferno.} irá se jogar bruscamente nos braços de um
velho que estiver passando ao seu lado, e o beijará com entusiasmo
ante a multidão espantada.

Por quê? Porque\ldots\  porque esta fisionomia lhe era irresistivelmente
simpática? Talvez; mas é mais legítimo supor que ele próprio não saiba o
por quê.

Fui vítima, mais de uma vez, destas crises e impulsos que nos
autorizam a crer que Demônios maliciosos se insinuam dentro de nós e
nos fazem cumprir, à revelia, suas mais absurdas vontades.

Certa manhã, levantei"-me aborrecido, triste, cansado de ociosidade e
levado, assim me pareceu, a efetuar algo grande, uma ação de brilho; e infelizmente, abri a
janela!

(Queiram, por favor, observar que o espírito de mistificação, que em
certas pessoas não é fruto de um trabalho ou de uma combinação, e sim 
de uma inspiração fortuita, tem


\noindent{}parte, muito, mesmo que apenas pelo
ardor do desejo, neste humor, histérico segundo os médicos, satânico
segundo os que pensam um pouco melhor que os médicos, que nos
impele sem resistência a uma porção de ações perigosas ou
inconvenientes.)

A primeira pessoa que avistei na rua foi um vidraceiro, cujo grito
penetrante, dissonante, chegou"-me através da pesada e suja atmosfera
parisiense. Seria, aliás, impossível dizer por que fui tomado, em
relação ao pobre homem, de um ódio tão repentino quanto despótico.

``Ei, ei'' e eu lhe gritei que subisse.
Entretanto eu refletia, não sem certa alegria, que, ficando o quarto
no sexto andar e sendo a escada bastante estreita, o
homem devia estar experimentando certa dificuldade em efetuar sua
ascensão, e esbarrando em diversos lugares os ângulos de sua frágil
mercadoria.

Ele enfim apareceu: examinei com curiosidade todas as suas vidraças e
lhe disse: ``Mas como? Você não tem vidros coloridos?
Vidros cor"-de"-rosa, vermelhos, azuis, vidros mágicos, vidros de
paraíso? Que atrevido é você! Ousa passear pelos bairros pobres e nem
sequer tem vidros que tornem a vida bela de se ver!'' E
o empurrei energicamente para a escada, na qual tropeçou, resmungando.

Aproximei"-me da sacada, agarrei um vasinho de flores e, quando o
homem reapareceu no vão da porta, deixei cair perpendicularmente meu
engenho de guerra na borda traseira das suas forquilhas. Desabando
com o choque, ele acabou de destroçar sob suas costas toda a sua pobre
fortuna ambulativa, que produziu o ruído estrondoso de um palácio de
cristal estraçalhado por um raio.

E embriagado por minha loucura, gritei"-lhe furiosamente:
``A vida bela de se ver! A vida bela de se
ver!''

Essas brincadeiras nervosas não são isentas de perigo, e pode"-se às
vezes pagar caro por elas. Mas o que importa a eternidade da danação a
quem encontrou num segundo o infinito da fruição?

\quebra\section[À uma hora da manhã]{à uma hora da manhã}

Enfim! Só! Já não se ouve mais que o movimento de alguns fiacres retardados
e estafados. Durante algumas horas, possuiremos o silêncio, se não o
repouso. Enfim! Desapareceu a tirania da face humana, e já não sofrerei
senão por mim mesmo.

Enfim! É-me então permitido repousar num banho de trevas! Primeiro, duas
voltas na fechadura. Parece"-me que girar a chave aumentará minha
solidão e fortificará as barricadas que me separam atualmente do mundo.

Vida horrível! Cidade horrível! Recapitulemos o dia: ter visto vários
homens de letras, um deles me tendo perguntado se era possível
ir à Rússia por via terrestre (vai ver, ele achava que a Rússia era
uma ilha); ter prodigamente discutido com o diretor de uma revista,
que a cada objeção respondia: --- ``Aqui é o partido das
pessoas honestas, o que implica em todos os outros
jornais serem redigidos por patifes; ter cumprimentado umas vinte
pessoas, quinze das quais me são desconhecidas; ter distribuído
apertos de mão na mesma proporção, e isto sem ter tomado a precaução de
comprar luvas; ter entrado, para matar o tempo durante um aguaceiro,
na casa de uma saltadora que me rogou lhe desenhasse um traje de \textit{
Vênis};\protect\footnote{  Vênis procura traduzir a palavra \textit{Vénustre} com que, em francês,
Baudelaire reproduz ironicamente a pronúncia da saltadora para Vênus.}~ter cortejado um diretor de teatro, que me
disse, ao dispensar"-me: --- “Talvez fosse bom você
dirigir"-se a Z\ldots\ ; é o mais pesado, o mais bobo e o mais famoso dos
meus autores, com o qual você talvez pudesse 

\quebra

\noindent{}chegar a alguma coisa. Vá
vê"-lo, e então veremos''; ter"-me gabado (por
quê?) de diversas más ações que nunca cometi, e ter covardemente negado
alguns outros malfeitos que cumpri com alegria, delito de fanfarronada,
crime de respeito humano; ter recusado a um amigo um fácil favor, e
dado uma recomendação por escrito a um perfeito malandro. Ufa! Será que
acabou?

Descontente com todos e descontente comigo, gostaria de me redimir
e me orgulhar um pouco, no silêncio e na solidão da noite. Almas
daqueles que amei, almas daqueles que cantei, fortaleçam"-me,
sustenham"-me, afastem de mim a mentira e os vapores corruptores do
mundo. E vós, Senhor, meu Deus! Concedei"-me a graça de produzir alguns
poucos versos belos, que provem a mim mesmo que não sou o último dos
homens, que não sou inferior àqueles que desprezo!

\quebra\section[A mulher selvagem e a pequena"-amante]{a mulher selvagem\break e a pequena"-amante}

“Você realmente, minha cara, me cansa sem medida e
sem dó. Até diria, quem a ouvisse suspirar, que você sofre mais que
as respigadeiras sexagenárias e as velhas mendigas que juntam cascas de
pão na porta das tabernas.

“Se ao menos seus suspiros expressassem remorsos, algum
jus lhe fariam; mas só traduzem a saciedade do bem"-estar e o
langor do repouso. Você, além disso, não cessa de derramar"-se em
palavras inúteis: ``Me ame muito! Preciso tanto! Me
console daqui! Me acaricie dali!'' Pois olhe!, quero tentar
curá"-la; a maneira, talvez a encontremos por dois tostões, em meio a
uma festa, e sem ir muito longe.

“Atentemos, lhe suplico, para esta sólida gaiola de
ferro dentro da qual se agita, berrando feito um condenado, sacudindo
as grades feito um orangotango exasperado pelo exílio, imitando com
perfeição ora os saltos circulares do tigre, ora os rebolados
estúpidos do urso branco, este monstro peludo cuja forma imita um tanto
vagamente a sua.

“Este monstro é um desses animais que chamamos geralmente
de ``meu anjo'', ou seja, uma mulher. O outro
monstro, o que grita desesperadamente com um bastão na mão, é um
marido. Ele acorrentou sua mulher legítima como um bicho e a exibe
pelos subúrbios nos dias de feira, com a permissão dos magistrados, é
escusado dizer.

\quebra

“Preste bem atenção! Veja com que voracidade (talvez não
simulada!) ela lacera os coelhos vivos e aves cacarejantes que lhe joga o
seu cornaca. ``Ora'', diz ele, ``não deve comer sua fortuna toda num só
dia'' e, com estas sábias palavras, arranca"-lhe
cruelmente sua presa, cujas tripas desenroladas ficam por um instante
penduradas nos dentes do animal feroz, quero dizer, da mulher.

“Ora, vamos! Uma pancada para acalmá"-la! Pois ela está fitando
seus terríveis olhos de cobiça no alimento tirado. Santo Deus! O bastão
não é um bastão de brincadeira! Você ouviu ressoar a carne, apesar do
pelo postiço? Assim, os olhos agora estão lhe saindo das órbitas, ela
está berrando ``mais naturalmente''. Em sua raiva, está faiscando
inteirinha, feito o ferro que se malha.

“Tais são os costumes conjugais desses dois descendentes
de Eva e Adão, essas obras de vossas mãos, oh!, meu Deus! Essa mulher
é incontestavelmente infeliz, se bem que talvez, afinal, os
gozos melindrosos da glória não lhe sejam desconhecidos. Existem
infelicidades mais irremediáveis e sem compensação. Mas no mundo em
que ela foi jogada, nunca teve como acreditar que a mulher merecesse
outro destino.

“Nós dois, agora, cara preciosa! Ao ver os infernos de que
o mundo está povoado, o que vai querer que eu pense do seu bonito
inferno, você que só descansa em tecidos suaves como sua pele,
só come carne cozida para quem um hábil criado toma o
cuidado de cortar os pedaços?

“E que significado podem ter para mim todos os pequenos
suspiros que enchem seu peito perfumado, robusta coquete? E todas essas
afetações aprendidas nos livros, e esta incansável melancolia feita
para inspirar no espectador um sentimento tão distinto da piedade? Na
verdade, às vezes sinto ganas de ensinar"-lhe o que é a real
desventura.

“Ver você assim, minha bela delicada, com os pés no lodo e
os olhos etereamente voltados para o céu como a pedir"-lhe um rei,
lembra genuinamente uma jovem rã a invocar o ideal. Embora
despreze a travezinha (que é o que sou agora, você bem sabe), tome
tento com o guindaste \textit{que há de trincá"-la, tragá"-la e matá"-la,}\protect\footnote{  Citação 
adaptada da fábula ``Les Grenouilles qui demandent un roi'', 
de La Fontaine (1621"-1695).}
\textit{a bel-prazer.}

“Por mais poeta que eu seja, não sou tão crédulo quanto
você gostaria de pensar, e se me cansar muito amiúde com seus \textit{preciosos}
choramingos, hei de tratá"-la como \textit{mulher selvagem}, ou jogá"-la pela
janela, como garrafa vazia.''

\quebra\section[As massas]{as massas}

Não é dado a qualquer um tomar banho de multidão. Desfrutar da massa é uma
arte e só poderá fazer, às custas do gênero humano, uma orgia de
vitalidade, aquele a quem uma fada terá insuflado no berço o gosto
pelo disfarce e a máscara, o ódio do domicílio e a paixão pela
viagem.

Multidão, solidão: termos iguais e permutáveis, para o poeta ativo e
fecundo. Quem não sabe povoar sua solidão tampouco sabe estar só em
meio a uma massa azafamada.

Goza o poeta desse incomparável privilégio de poder ser, a bel-prazer,
ele próprio e outrem. Igual a essas almas errantes em busca de um corpo,
ele entra, quando quer, na personagem de qualquer um. Para ele apenas, tudo
está vacante; e se alguns lugares lhe parecem estar fechados, é que a
seus olhos não valem a pena ser visitados.

O andarilho solitário e pensativo tira uma embriaguez singular desta
universal comunhão. Quem desposa facilmente a massa conhece gozos
febris, dos quais serão eternamente privados o egoísta, trancado como
um cofre, e o preguiçoso, internado como um molusco. Ele adota como
suas todas as profissões, todas as alegrias e todas as misérias que a
circunstância lhe apresenta.

O que os homens denominam amor é bem pequeno, restrito e frágil, se 
comparado a esta inefável orgia, a esta santa prostituição da alma
que se dá por inteiro, poesia e caridade, ao imprevisto que se mostra, ao
desconhecido que passa.

\quebra
É bom ensinar, às vezes, aos venturosos deste mundo, mesmo que só para
humilhar por um instante seu orgulho tolo, que existem venturas
superiores às suas, mais amplas e refinadas. Os fundadores de colônias,
os pastores de povos, os padres missionários exilados no fim do mundo,
decerto conhecem algo destas misteriosas embriaguezes; e, no seio da
vasta família que seu gênio construiu para si, eles por vezes devem rir
dos que se compadecem com sua sina tão agitada e sua vida tão
casta.

\quebra\section[As viúvas]{as viúvas}

Diz Vauvenargues\protect\footnote{
 Luc de Clapier, marquês de Vauvenargues (1715"-1747), moralista do
século \textsc{xviii}. Sua obra completa havia sido editada em 1857.}
 que nos jardins públicos há
alamedas frequentadas sobretudo pela ambição frustrada, pelos
inventores frustrados, as glórias abortadas, os corações partidos,
por todas essas almas tumultuosas e fechadas em que ainda trovejam
os últimos suspiros de uma tormenta, e que recuam para longe do olhar
insolente dos faceiros e ociosos. Esses retiros sombrios são os
pontos de encontro dos estropiados da vida.

É de preferência para esses lugares que o poeta e o filósofo gostam de
dirigir suas ávidas conjeturas. Há neles um alimento certo. Pois se
existe um lugar que eles desdenham visitar, como eu há pouco insinuava,
é sobretudo a alegria dos ricos. Tal turbulência no vazio nada tem que
os atraia. Pelo contrário, sentem-se irresistivelmente impelidos
por tudo aquilo que é órfão, frágil, arruinado, entristecido.

Um olhar experimentado nunca se engana. Nos traços tensos ou
abatidos, nos olhos fundos e apagados ou luzindo os derradeiros clarões
da luta, nas rugas profundas e várias, nos andares tão
lentos ou sôfregos, ele logo decifra as incontáveis lendas do amor
traído, da afeição desagradecida, dos esforços não recompensados, da
fome e do frio humilde, silenciosamente suportados.

Você alguma vez, nesses bancos solitários, avistou viúvas, viúvas
pobres? Que estejam, ou não, usando luto, é fácil reconhecê"-las. Sempre existe, aliás, no luto do pobre alguma coisa que falta, uma ausência de
harmonia que o torna mais aflitivo. O pobre é obrigado a regatear sua dor. O
rico ostenta a sua em grande escala.

Qual a viúva mais triste e mais entristecedora, a que traz pela mão
um pimpolho com quem não pode partilhar seu devaneio, ou aquela que
está totalmente só? Não sei\ldots\  Aconteceu"-me uma vez seguir por
longas horas uma velha aflita dessa espécie; ela, tesa, tensa, sob um
xale surradinho, trazia em todo o seu ser o orgulho dos estoicos.

Estava evidentemente condenada, por absoluta solidão, a hábitos
de velho celibatário, e o caráter masculino de seus modos 
juntava à sua austeridade um misterioso sabor. Não sei em que
bar miserável e de que jeito ela almoçou. Eu a segui até o gabinete de
leitura, e me demorei a espiar enquanto ela buscava nas gazetas,
com olhos ativos, outrora queimados pelas lágrimas, notícias de um
interesse possante e pessoal.

À tarde, enfim, sob um encantador céu de outono, desses céus de que chovem multidões de saudades e recordações, sentou"-se à parte num
jardim para escutar, longe da multidão, um desses concertos cuja
música de regimento gratifica o povo parisiense.

Esta era, decerto, a devassidão daquela velha inocente (ou daquela velha
purificada), o consolo bem merecido de um desses pesados dias sem
amigo, sem conversa, sem alegria, sem confidente, que Deus deitava 
sobre ela, desde

\quebra

\noindent{}vários anos talvez, trezentas e sessenta e cinco vezes ao ano.

Outra ainda:

Nunca consigo evitar dar uma olhada, se não universalmente simpática,
pelo menos curiosa, na multidão de párias que se amontoam junto ao
cercado de um concerto público. A orquestra joga noite adentro cantos de festa, triunfo ou volúpia. Os vestidos arrastam pelo chão, reverberando; os
olhares se cruzam; os ociosos, cansados de nada terem feito, bamboleiam fingindo degustar indolentemente a música. Aqui, nada que
não seja rico, feliz; nada que não respire e inspire a despreocupação e o
prazer de deixar"-se viver;\protect\footnote{ O pintor Edouard Manet (1832"-1883) tornara"-se, por volta de 1860,
amigo íntimo de Baudelaire. Seu quadro \textit{La musique aux Tuileries}
(pintado em 1861, exposto em 1862), nascido de uma sugestão do poeta,
aproxima"-se desta descrição.} nada exceto o aspecto
daquela turba encostada lá adiante, na cerca externa, recolhendo de graça,
ao sabor do vento, um retalho de música, e fitando a resplandecente
fornalha interna.

Sempre é interessante este reflexo da alegria do rico no fundo dos
olhos do pobre. Naquele dia, porém, em meio ao povo vestido de aventais
e de chita, avistei uma criatura cuja nobreza causava vistoso contraste com
toda a trivialidade em redor.

Era uma mulher alta, majestosa, e tão nobre em todo o seu porte que não
lembro ter visto alguma que a igualasse no elenco das
aristocráticas beldades do passado. Um perfume de altiva virtude
emanava de toda a sua pessoa. Seu rosto, triste e emagrecido, estava em
plena concordância com o luto cerrado que a revestia. Também ela,
como a plebe à qual se misturava e que não via, fitava o mundo
luminoso com um olhar profundo e escutava meneando suavemente a
cabeça.

Visão singular! “Seguramente,'' pensei, “essa pobreza,
se pobreza houver, não deve tolerar a economia sórdida; um rosto tão
nobre o garante. Por que será que fica voluntariamente num ambiente em
que destoa tão visivelmente?''

Mas passando, curioso, perto dela, julguei adivinhar o motivo. A alta
viúva segurava pela mão uma criança, vestida de preto como ela; por
módico que fosse o preço da entrada, este preço talvez bastasse para
pagar uma necessidade da criaturinha, ou melhor ainda, uma
superfluidade, um brinquedo.

E ela terá voltado a pé para casa, meditando e sonhando, só, sempre só;
pois a criança é turbulenta, egoísta, sem doçura e sem paciência, e nem
sequer pode, como o puro animal, como o cão e o gato, servir de
confidente aos sofrimentos solitários.

\quebra\section[O velho saltimbanco]{o velho saltimbanco}

Por toda parte se espalhava se espraiava, se recreava o povo em férias.
Era uma dessas solenidades com que contam, por um bom tempo, os
saltimbancos, fazedores de truques, mostradores de animais e 
vendedores ambulantes, para compensar as más temporadas do ano.

Nesses dias, tenho impressão que o povo esquece tudo, a dor e o trabalho;
torna"-se igual às crianças. Para os pequenos é um dia feriado, é o
horror da escola remetido a vinte e quatro horas. Para a gente grande é
um armistício concluído com as forças malignas da vida, uma trégua na
contenda e na luta universais.

O próprio homem mundano, e o homem ocupado com trabalhos espirituais, dificilmente escapam da influência deste jubileu popular. 
Absorvem, sem querer, a parte que lhes cabe desta atmosfera de
despreocupação. Quanto a mim, como legítimo parisiense, nunca deixo
de passar em revista todas as barracas que se exibem nessas épocas
solenes.

Elas travavam, na verdade, um concorrência formidável: piavam, bramiam,
uivavam. Era uma mescla de gritos, detonações de cobre e explosões de
foguetes. Palhaços e farsistas\protect\footnote{ Palhaços e farsistas procuram 
traduzir \textit{queues rouges} (palhaços
portadores de uma peruca vermelha em forma de rabo"-de"-cavalo) e
\textit{jocrisses} (valetes, personagens de comédia).} convulsavam as feições
de seus rostos queimados, enrugados pelo vento, pela chuva e pelo sol;
lançavam, com o aprumo dos comediantes seguros do efeito que causam,
gracejos e brincadeiras de uma comicidade forte e pesada, como a de
Molière. Os Hércules, orgulhosos da enormidade de seus membros, sem
testa e sem crânio como orangotangos, se espreguiçavam
majestosamente dentro de malhas lavadas na véspera para a circunstância. As
dançarinas, lindas como fadas ou princesas, saltavam e cabriolavam à
chama das lanternas que enchia suas saias de faíscas.

Tudo era apenas luz, poeira, gritos, alegria, tumulto; uns gastavam,
outros ganhavam, uns e outros igualmente alegres. As crianças
penduravam"-se na saia das mães para pedir algum pirulito, ou
subiam nos ombros dos pais para melhor enxergar um ilusionista
deslumbrante feito um deus. E por toda parte circulava, dominando todos
os aromas, um cheiro de fritura que era como o incenso desta festa.

Na ponta, bem na ponta da fileira de barracas, como se,
envergonhado, tivesse exilado a si próprio daqueles esplendores todos,
avistei um pobre saltimbanco, encurvado, caduco, decrépito, uma
ruína humana, encostado num dos mourões de sua palhoça, palhoça
mais miserável que a do mais embrutecido selvagem, da qual dois tocos
de vela, escorrendo e fumegando, ainda iluminavam bem demais o
desespero.

Por toda parte a alegria, o ganho, a devassidão; por toda parte a
certeza do pão dos amanhãs; por toda parte a explosão frenética da
vitalidade. Aqui, a miséria absoluta, a miséria revestida, para cúmulo
do horror, de andrajos cômicos, nos quais a necessidade, bem mais do
que a arte, introduzira o contraste. Ele não ria, o miserável! Não
chorava, não dançava, não gesticulava, não gritava; não cantava canção nenhuma, alegre ou lamentável, não implorava. Estava parado e
mudo. Ele renunciara, abdicara. Seu destino estava
dado.

Mas que olhar profundo, inesquecível, ele espraiava na multidão e
nas luzes, cuja vaga movediça detinha"-se a poucos passos de sua repulsiva
miséria! Senti minha garganta apertar"-se na mão terrível da histeria,
e pareceu que meus olhares se ofuscavam com essas lágrimas
rebeldes que não querem cair.

Que fazer? De que serviria perguntar ao infeliz qual a curiosidade,
a maravilha que ele tinha para mostrar naquelas trevas malcheirosas,
detrás de sua cortina lacerada? Na verdade, eu não ousava; e, tivesse
de os fazer rir a razão de minha timidez, confesso que temia
humilhá"-lo. Afinal, estava decidindo deixar, ao passar, algum
dinheiro sobre uma das tábuas, esperando que ele percebesse minha
intenção, quando um grande refluxo de povo, causado por não sei que
confusão, arrastou"-me para longe dali.

E, indo embora, obcecado por aquela visão, tentei analisar minha
dor repentina e pensei comigo mesmo: acabo de ver a imagem do velho homem de
letras que sobreviveu à sua geração, de que foi o brilhante animador;
do velho poeta sem amigos, sem família, sem filhos, degradado por sua
miséria e pela ingratidão do público, e em cuja barraca o mundo esquecido
não quer mais entrar.

\quebra\section[O bolo]{o bolo}

Eu viajava. A paisagem na qual eu me encontrava era de uma grandeza e uma
nobreza irresistíveis. Algo delas decerto passou naquele momento
para minha alma. Meus pensamentos esvoaçavam com uma leveza igual à da
atmosfera; as paixões vulgares, como o ódio e o amor profano, me
pareciam agora tão distantes como as névoas que resvalavam no fundo
do abismo aos meus pés; minha alma me parecia tão vasta e tão pura
como a cúpula do céu que me envolvia; e a lembrança das
coisas terrestres só me chegava ao coração enfraquecida e diminuída,
como o som da sineta dos gados imperceptíveis que passavam longe, bem
longe, na vertente de outra montanha. Sobre o pequeno lago imóvel,
negro em sua imensa profundez, passava às vezes a sombra de uma
nuvem, qual reflexo do manto de um gigante aéreo a voar pelo céu. E
lembro que esta sensação solene e rara, causada por um vasto movimento
perfeitamente silente, me enchia de uma alegria entremeada de medo.
Em suma, eu me sentia, graças à entusiasmante beleza de que estava
cercado, em perfeita paz comigo e com o universo; acho até que, em minha
perfeita beatitude e em meu total esquecimento de todo o mal terrestre,
chegara a não mais julgar tão ridículos os jornais que afirmam que o
homem nasceu bom --- quando, a matéria incurável renovando suas
exigências, pensei em reparar o cansaço e aliviar o apetite causados
por tão longa ascensão. Puxei do bolso um pedaço grande de pão, uma
xícara de couro e um

\quebra

\noindent{}frasco de um certo elixir que os farmacêuticos
vendiam naquele tempo aos turistas para ser oportunamente misturado com
água de neve.

Eu cortava tranquilamente meu pão, quando um levíssimo ruído me fez
erguer os olhos. Em pé na minha frente estava um serzinho andrajoso,
negro, desgrenhado, cujos olhos cavos, ferozes e como que suplicantes,
devoravam o pedaço de pão. E eu o ouvi suspirar, com uma voz baixa e
rouca, a palavra: \textit{bolo}! Não pude deixar de rir ao escutar a apelação
com que ele tinha a bondade de honrar meu pão quase branco, e cortei
para ele uma boa fatia que lhe ofereci. Devagar, ele se acercou,
sem tirar os olhos do objeto da sua cobiça; então, agarrando o pedaço,
recuou depressa, como temendo que minha oferta não fosse sincera ou
que eu já me estivesse arrependendo.

Mas, naquele instante, foi derrubado por outro selvagenzinho, surgido de
não sei onde, e tão absolutamente parecido com o primeiro que poderia
ser tomado por seu irmão gêmeo. Juntos rolaram pelo chão, disputando a
preciosa presa, nenhum deles querendo, decerto, renunciar à metade
pelo irmão. O primeiro, exasperado, empunhou o segundo pelos cabelos;
este lhe agarrou a orelha com os dentes, cuspindo um pedacinho
sangrento junto com uma fantástica praga em gíria. O legítimo proprietário do
bolo tentou enfiar suas garrinhas nos olhos do usurpador; este,
por sua vez, empregou toda a sua força para estrangular seu
adversário com uma das mãos, enquanto com a outra tentava enfiar no
bolso o prêmio do combate. Mas, reatiçado pelo desespero, o vencido
se reergueu e fez rolar por terra o vencedor com uma cabeçada no
estômago. De que serviria descrever uma luta hedionda que, em verdade,
durou mais tempo do que pareciam prometer suas forças infantis? O bolo
viajava de mão em mão e mudava de bolso a todo instante; mas, ai!,
mudava também de volume; e quando afinal, extenuados, ofegantes,
ensanguentados, pararam por impossibilidade de continuar, já não
havia, a bem dizer, nenhum motivo para batalha: o pedaço de pão sumira, e
estava disperso em farelos semelhantes aos grãos de areia a que se
misturava.

Aquela cena me tinha enevoado a paisagem, e a alegria calma em que
se recreava minha alma antes de ver os homenzinhos
desaparecera totalmente. Fiquei um bom tempo triste, me repetindo sem
cessar: ``Com que então existe uma terra fantástica, onde o pão
se chama \textit{bolo}, iguaria tão rara que é o bastante para gerar uma
guerra perfeitamente fratricida!''

\quebra\section[O relógio]{o relógio}

Os chineses veem as horas no olho dos gatos

Um missionário, passeando um dia pela periferia de Nanquim, percebeu que
tinha esquecido o relógio, e perguntou a um menino que horas eram.

O garoto do celeste Império primeiro hesitou; depois, reconsiderando,
respondeu: ``Já vou lhe dizer''. Passados
alguns instantes, voltou trazendo no colo um gato bem grande, e
olhando para ele, para dentro dos seus olhos, como se diz, afirmou sem
hesitar: ``Ainda não é bem meio"-dia'' --- o
que era verdade.

Quanto a mim, se me debruço sobre a bela Felina, a tão bem denominada,
que é a um só tempo a honra do seu sexo, o orgulho do meu coração e o
perfume do meu espírito, quer de noite, quer de dia, em plena luz ou na
sombra opaca, no fundo de seus olhos adoráveis sempre vejo
distintamente a hora, sempre a mesma, uma hora vasta, solene, do tamanho do espaço, sem divisões em minutos ou segundos --- uma hora imóvel
não marcada nos relógios, porém leve como um suspiro,
veloz como uma espiada.

E se algum importuno me viesse perturbar, estando o meu olhar
repousando neste gracioso mostrador, se algum Gênio desonesto e
intolerante, algum Demônio do contratempo me viesse dizer:
``O que está mirando com tanto cuidado? O que está
buscando nos olhos deste ser? Você neles vê as horas, mortal pródigo
e vadio?'' eu responderia sem hesitar: 

``Sim,
vejo as horas; são a Eternidade''.

Não será este madrigal, madame, realmente meritório, e tão enfático
quanto a senhora? Em verdade, tal foi meu prazer em tecer este
pretensioso galanteio, que nada em troca hei de pedir"-lhe.

\quebra\section[Um hemisfério numa cabeleira]{um hemisfério numa cabeleira}

Me deixe respirar, por longo, longo tempo, o cheiro dos seus cabelos,
mergulhar neles meu rosto inteiro como um homem sedento na água de uma
fonte, e agitá"-los com a mão como a um lenço cheiroso, para
sacudir lembranças no ar.

Se você soubese tudo o que vejo! Tudo o que sinto! Tudo o que escuto
em seus cabelos! Minha alma viaja pelo perfume como a de
outros homens viaja pela música.

Seus cabelos contêm todo um sonho, repleto de velas e mastros; contêm
vastos mares cujas monções me conduzem a encantadoras regiões, onde o
espaço é mais azul e mais profundo, onde a atmosfera é perfumada pelas
frutas, pelas folhas e pela pele humana.

No oceano de sua cabeleira, entrevejo um porto fervilhando de cantos
melancólicos, homens vigorosos de todas as nações, e navios de todas as
formas recortando suas finas e complicadas estruturas num céu imenso,
onde se estira o calor eterno.

Nas carícias da sua cabeleira, revivo os langores de longas horas
passadas num sofá, no quarto de um belo navio, embaladas pela
imperceptível arfagem do porto, entre vasos de flores e moringas
refrescantes.

Na ardente lareira da sua cabeleira, respiro o cheiro do fumo mesclado
de ópio e açúcar; na noite da sua cabeleira, vejo refulgir o infinito
do azul tropical; nas margens de \linebreak

\quebra

\noindent{}penugem da sua cabeleira, me embriago
com os cheiros combinados de alcatrão, almíscar e óleo de coco.

Me deixe morder, por longo tempo, suas tranças pesadas e negras. Quando
mordisco seus cabelos maleáveis e rebeldes, me sinto comendo
lembranças.

\quebra\section[Convite à viagem]{convite à viagem}

Existe uma terra esplêndida, uma terra de promissão,\protect\footnote{ 
 A descrição desta terra de promissão sugere a Holanda, à qual também
se referem vários detalhes do poema, notadamente no 6º parágrafo.}
é o que dizem, que eu sonho em visitar com uma velha amiga. Terra
singular, imersa nas brumas do nosso norte, e que poderíamos chamar de
Oriente do Ocidente, de China da Europa, tanto ali se deu asas a
quente e caprichosa fantasia, tanto a ilustrou, paciente e
teimosamente, com suas vegetações delicadas e sábias.

Verdadeira terra de promissão, onde tudo é belo, rico, tranquilo,
honesto; onde o luxo sente prazer em mirar"-se na ordem; onde a vida é
grassa e doce de respirar; onde a desordem, a turbulência e o
imprevisto estão excluídos; onde a felicidade está casada com o
silêncio; onde a própria comida é poética, a um
só tempo gordurosa e excitante; onde tudo se parece com você, meu caro anjo.

Você conhece essa doença febril que toma conta de nós nas frias
misérias, essa nostalgia da terra que ignoramos, essa angústia da
curiosidade? Existe um lugar que se parece com você, onde tudo é
belo, rico, honesto e tranquilo, onde a fantasia construiu e decorou
uma China ocidental, onde a vida é doce de se respirar, onde a
felicidade está casada com o silêncio. É lá que é preciso ir viver, é
lá que é preciso ir morrer!

Sim, é lá que é preciso ir respirar, sonhar e alongar as horas pelo
infinito das sensações. Um músico escreveu \textit{O convite à
valsa},\protect\footnote{ Obra do compositor alemão Carl von Weber (1786"-1826).}
quem é que irá compor \textit{O convite à
viagem}, que se possa ofertar à mulher amada, à irmã
dileta?\protect\footnote{ A sugestão de Baudelaire foi posteriormente posta em prática por
vários compositores: surgiriam seis \textit{Convite à Viagem}, sendo a de Duparc (1870) a
mais conhecida.}

Sim, é nesta atmosfera que seria bom viver --- lá, onde as horas mais
lentas contêm mais pensamentos, onde os relógios batem a felicidade
com solenidade mais profunda e expressiva.

Em painéis cintilantes, ou em couros dourados e de sombria riqueza,
vivem discretamente pinturas beatas, profundas e calmas como as almas
dos artistas que as criaram. Os sóis poentes, que tão ricamente colorem
a sala de jantar ou de visitas, são filtrados por lindos tecidos ou por
essas altas janelas lavradas que o chumbo comparte em diversas
divisões. Os móveis são amplos, curiosos, estranhos, dotados de
fechaduras e segredos como almas refinadas. Os espelhos, os metais, os
tecidos, a ourivesaria e a faiança executam para os olhos uma
silente e misteriosa sinfonia; e de todas as coisas, de todos os recantos, das
fissuras das gavetas e das pregas dos tecidos, desprende-se um aroma
singular, \textit{reminiscência} de Sumatra, que é como a alma do apartamento.

Verdadeira terra de promissão, estou dizendo, onde tudo é rico,
limpo e luzente, como uma bela consciência, como suntuosa bateria
de cozinha, como esplêndida ourivesaria, como joia colorida!
Ali os tesouros do mundo afluem, como na casa do homem laborioso
que bem mereceu do mundo inteiro. País singular, superior aos demais,
como é a Arte superior à Natureza, quando esta é reformada pelo sonho,
é corrigida, embelezada, refundida.

Que busquem, continuem buscando, que estendam sem cessar os limites de
sua felicidade, estes alquimistas da horticultura! Que proponham
prêmios de setenta e cem mil florins a quem resolver seus
ambiciosos problemas! Quanto a mim, encontrei minha \textit{tulipa negra} e
minha \textit{dália azul}!

Flor incomparável, tulipa reencontrada, alegórica dália, não é lá,
nesta terra linda, tão calma e sonhadora, que seria preciso ir viver e
florescer? Você não se enquadraria com a sua analogia, e não se
poderia mirar, para falar como falam os místicos, em sua própria
\textit{correspondência}?

Sonhos! Sempre sonhos! E quanto mais ambiciosa e delicada é a alma, mais
os sonhos a afastam do possível. Todo homem traz em si sua dose de ópio
natural, incessantemente secretada e renovada e, do nascimento à
morte, quantas horas contamos preenchidas de gozo positivo, de
ação bem"-sucedida e decidida? Será que ainda iremos viver, passar algum dia
para este quadro que meu espírito pintou, este quadro que se assemelha
você?

Estes tesouros, móveis, este luxo, esta ordem, os aromas, as
flores milagrosas, são você. Também são você esses grandes rios e
canais tranquilos. Esses enormes navios que eles levam, todos
repletos de riquezas e de onde se elevam os monótonos cantos da
manobra, são meus pensamentos que dormem ou rolam sobre o seu seio.
Você os conduz suavemente para o mar, que é o Infinito, refletindo as
profundezas do céu na limpidez de sua alma linda --- e quando,
cansados do marulho e cevados dos produtos do Oriente, eles voltam ao
porto natal, são também meus pensamentos enriquecidos voltando do
Infinito a você.

\quebra\section[O brinquedo do pobre]{o brinquedo do pobre}

Quero dar a ideia de um divertimento inocente. Há tão poucas diversões
que não sejam culposas! Quando sair pela manhã com a resoluta
intenção de vagar pelas estradas, encha seus bolsos de miúdos
inventos de um tostão --- tais como o simples polichinelo puxado por um
só fio, os ferreiros que batem a bigorna, o cavaleiro com seu cavalo,
cuja cauda é um apito --- e pelas tabernas, embaixo das
árvores, faça uma oferenda às crianças desconhecidas e pobres que
encontrar. Vai ver como arregalam extraordinariamente os olhos.
Primeiro, não vão se atrever a tocá-lo; vão duvidar da própria sorte. Depois, vão agarrar avidamente o presente e fugir, como fazem os
gatos que vão comer longe de você o bocado que você lhes
deu, tendo aprendido a desconfiar do homem.

Numa estrada, atrás do portão gradeado de um vasto jardim, ao fundo do qual aparecia
a brancura de um belo castelo fustigado pelo sol, estava uma
criança bonita e viçosa, trajando essas roupas campestres de tanta
faceirice.

O luxo, a despreocupação e a visão habitual da riqueza tornam essas
crianças tão bonitas que parecem ter sido moldadas numa massa distinta da
dos filhos da mediocridade ou da pobreza.

A seu lado, jazia na relva um brinquedo maravilhoso, tão viçoso
quanto o dono, envernizado, dourado, vestindo uma roupa púrpura
e coberto de plumas e miçangas. Mas a criança não dava atenção ao seu
brinquedo preferido, e eis o que ela olhava:

Do outro lado da cerca, na estrada, entre os cardos e as urtigas, estava
outra criança, suja, raquítica, fuliginosa, um desses moleques"-párias
de que um olhar imparcial descobriria a beleza se, assim como o olhar do
entendido intui uma pintura ideal sob um verniz de segeiro,
removesse a pátina repulsiva da miséria.

Através dessas grades simbólicas que apartam dois mundos, a estrada e o
castelo, a criança pobre mostrava à criança rica o seu próprio
brinquedo, que esta examinava avidamente como a um objeto raro e
ignorado. Ora, o tal brinquedo, que o moleque sujinho atiçava,
agitava e chacoalhava numa caixa gradeada, era um rato vivo! Os pais,
por economia decerto, tinham tirado o brinquedo da própria vida.

E as duas crianças riam fraternalmente uma para a outra, com dentes de
\textit{igual} brancura.

\quebra\section[Os dons das fadas]{os dons das fadas}

Era uma grande assembleia das fadas, para proceder à partilha dos dons
entre todos os recém"-nascidos, vindos à vida desde vinte e quatro horas.

Aquelas antigas e caprichosas Irmãs do Destino,
estranhas Mães da alegria e da dor, eram todas bastante distintas: umas
tinham um ar sombrio e carrancudo; outras, um ar faceiro e esperto;
umas, jovens que sempre tinham sido jovens; outras, velhas que sempre
tinham sido velhas.

Todos os pais que têm fé nas Fadas tinham vindo, trazendo cada qual seu
recém"-nascido nos braços.

Os Dons, as Faculdades, os bons Acasos, as Circunstâncias invencíveis,
estavam acumulados ao lado do tribunal, como os prêmios sobre o estrado
numa festa de fim de ano letivo. O que havia ali de singular é que os
Dons não eram a recompensa por algum esforço mas, muito pelo contrário,
uma graça concedida àquele que ainda não vivera, uma graça que podia
determinar seu destino e tanto se tornar fonte de sua infelicidade
como de sua felicidade.

As pobres Fadas estavam muito atarefadas, pois a massa dos solicitantes
era grande e o mundo intermediário, situado entre o homem e Deus, está
sujeito, como nós, à terrível lei do Tempo e sua infinita
posteridade, os Dias, as Horas, os Minutos, os Segundos.

Na verdade, estavam tão atordoadas como ministros em dia de
audiência ou funcionários do \textit{Mont"-de"-Piété}\protect\footnote{ Mont"-de"-Piété: Casa de Penhores.}
quando uma festa nacional autoriza os desempenhos gratuitos. \linebreak


\noindent{}Acho até que, vez ou outra, olhavam para o ponteiro do relógio com a mesma
impaciência com que juízes humanos, em audiência desde cedo, não
podem evitar de sonhar com o jantar, a família e seus caros
chinelos. Se existe, na justiça sobrenatural, um pouco de precipitação
e de acaso, não estranhemos que o mesmo se dê às vezes na justiça
humana. Ou nós mesmos é que estaremos sendo juízes injustos.

De modo que foram cometidos neste dia uns disparates que poderíamos
considerar como estranhos, se a prudência, mais que o capricho, fosse o
caráter distintivo, eterno, das Fadas.

Assim é que o poder de atrair magneticamente a riqueza foi adjudicado ao
herdeiro único de uma família muito rica, o qual, não tendo sido dotado
de nenhum senso de caridade, nem tampouco de cobiça pelos bens
mais visíveis da vida, haveria de se ver mais tarde tremendamente
embaraçado com seus milhões.

Assim é que foram dados o amor da Beleza e o Poder poético ao filho de um
obscuro indigente, canteiro de profissão, que não podia, de forma
alguma, estimular as faculdades nem suprir as necessidades de sua
deplorável descendência.

Esqueci de dizer que a distribuição, nestes casos solenes, é sem
apelação, e nenhum dom pode ser recusado.

Todas as fadas iam se levantando, pensando estar cumprida a tarefa
– pois não restava nenhum presente, nenhuma dádiva a jogar àquela escória humana –, quando um bom homem, acho que um pobre
comerciantezinho, levantou"-se e, segurando o vestido de vapores
multicolores da Fada que estava mais a seu alcance, exclamou:

“Ei, Senhora! Estão esquecendo de nós! Ainda falta
o meu menino! Não quero ter vindo para nada\ldots\ ”

A Fada bem que podia ficar embaraçada; pois não restava mais \textit{nada}. No
entanto, lembrou a tempo de uma lei bem conhecida, se bem que raramente
aplicada, do mundo sobrenatural, habitado por estas deidades
impalpáveis, amigas do homem e muitas vezes forçadas a se adaptarem a
suas paixões, tais como as Fadas, Gnomos, Salamandras,
Sílfides, Silfos, Nixas, Ondinos,\protect\footnote{ Os Gnomos são espíritos da terra, as Salamandras do fogo, as Sílfides
e os Silfos, dos ares, os Ondinos e as Ondinas (Nixas em alemão), das
águas, segundo mitologias diversas.} quero dizer, a lei que concede às Fadas em
caso semelhante, ou seja, de esgotamento dos prêmios, a
faculdade de ainda dar mais um, suplementar e excepcional, desde que todavia
ela tenha imaginação suficiente para criá"-lo de imediato.

De modo que a boa Fada respondeu, com um aprumo digno de sua classe:
``Dou a teu filho\ldots\  eu lhe dou\ldots\  o \textit{Dom de
agradar}!''

``Mas agradar como? Agradar\ldots\ ? Agradar por quê? perguntou insistentemente 
o pequeno lojista, que era decerto um desses
raciocinadores tão comuns, incapazes de se alçar à lógica do
Absurdo.''

``Porque sim! Porque sim!'', replicou a Fada,
irritada, dando"-lhe as costas; e, juntando"-se ao cortejo de suas
companheiras, dizia"-lhes: ``O que acham deste
francezinho vaidoso que quer entender tudo e, tendo obtido para o filho o 

\quebra

\noindent{}melhor de todos os prêmios, ainda se atreve a questionar e discutir o
Indiscutível?''

\quebra\section[As tentações, ou eros, pluto e a grória]{as tentações, ou~eros,~pluto\protect\footnote{\uppercase{N}a mitologia 
grega, \uppercase{E}ros era o deus do amor e do desejo, e \uppercase{P}luto, o deus das riquezas.}~e~a~glória}

Dois magníficos Satãs e uma Diaba não menos extraordinária subiram,
noite passada, a escada misteriosa por onde o Inferno investe contra a
fraqueza do homem que dorme e se comunica secretamente com ele. E 
vieram gloriosamente postar"-se diante de mim, em pé como sobre um
estrado. Um esplendor sulfuroso emanava dos três personagens, que
assim se destacavam do fundo opaco da noite. Tinham um jeito tão altivo
e cheio de dominação, que de início os tomei, os três, por autênticos Deuses.

O rosto do primeiro Satã era de sexo ambíguo, e havia também, nas linhas
de seu corpo, a brandura de antigos Bacos.\protect\footnote{ Baco era o deus 
romano do vinho (Dioniso, na mitologia grega).}
Seus belos olhos lânguidos, de cor tenebrosa e indecisa, lembravam
violetas ainda carregadas dos prantos pesados da tormenta, e seus
lábios entreabertos, caçoulas quentes que exalavam o cheiro
bom de uma perfumaria. Toda vez que ele suspirava, insetos
almiscarados se iluminavam, esvoaçando, com o ardor do seu sopro.

Em torno à sua túnica de púrpura se enrolava, à maneira de um cinto,
uma serpente cintilante que, cabeça erguida, voltava
languidamente para ele seus olhos de brasa. Neste cinto vivo pendiam, em alternância com frasquinhos cheios de sinistros licores,
facas brilhantes e instrumentos cirúrgicos. Na mão direita, ele segurava
outro frasquinho, de um conteúdo vermelho luminoso e que
tinha como rótulo esses estranhos dizeres: ``Bebei,
isto é o meu sangue, um perfeito

\quebra

\noindent{}cordial''; na esquerda,
um violino que decerto lhe servia para cantar seus prazeres e
dores, e para espalhar o contágio da sua loucura nas noites de
Sabá.\protect\footnote{ As noites de Sabá eram assembleias de bruxos e bruxas, segundo a
superstição popular.}

Em seus tornozelos delicados se arrastavam alguns elos de uma corrente
de ouro partida e, quando o incômodo em que resultavam o forçava a
voltar os olhos para o chão, contemplava vaidosamente as unhas de
seus pés, brilhantes e polidas feito pedras bem trabalhadas.

Mirou-me com seus olhos inconsolavelmente magoados, que
escorriam uma insidiosa embriaguez, e disse com voz melodiosa:
“Se você quiser, se quiser, hei de torná"-lo senhor das
almas e você será o mestre da matéria viva, mais ainda que o
escultor pode ser da argila; e você conhecerá o prazer,
continuamente renovado, de sair de si mesmo para esquecer"-se em
outrem, e de atrair as outras almas até confundi"-las com a
sua”.

E eu lhe respondi: “Muitíssimo obrigado! Não me interessa
esta pacotilha de seres que, sem dúvida, não valem mais que meu
pobre eu. Embora sinta alguma vergonha em relembrar, nada quero
esquecer. E mesmo que não o conhecesse, velho monstro, sua misteriosa
cutelaria, seus frasquinhos equívocos, as correntes em que seus pés se
enredam são símbolos que explicam com suficiente clareza os
inconvenientes da sua amizade. Fique com seus presentes”.

O segundo Satã não tinha nem aquele jeito ao mesmo tempo trágico e
sorridente, nem aquelas belas maneiras insinuantes, nem aquela beleza
delicada e perfumada. Era um homem amplo, de largo rosto sem olhos,
cuja pança pesada sobressaía sobre as coxas e cuja pele era
dourada e ilustrada com uma série de figurinhas
moventes que representavam, feito tatuagens, as formas várias da miséria universal. Eram
homenzinhos esguios que se suspendiam voluntariamente a um prego; Eram
pequenos gnomos disformes, magros, cujos olhos suplicantes imploravam
esmola melhor que suas mãos trêmulas; e também velhas mães
carregando rebentos grudados nas mamas extenuadas. Eram muitas coisas
mais.

O grande Satã batia com o punho na imensa barriga, da qual irrompia
então um longo e ecoante tinido metálico que terminava num vago
gemido feito de várias vozes humanas. E ele ria mostrando,
impudente, os dentes estragados com um enorme riso imbecil, como
alguns homens de qualquer país quando jantaram bem demais.

E este me disse: ``Posso lhe dar aquilo que tudo
consegue, tudo vale, tudo substitui!'' E bateu na
monstruosa barriga, cujo eco sonoro fez o comentário do seu grosseiro
palavreado.

Virei o rosto, enojado, e respondi: ``Não preciso, para o meu
prazer, da miséria de ninguém; e não quero uma riqueza entristecida, feito
papel de parede, por todas as desgraças representadas em sua
pele''.

Quanto à Diaba, mentiria se não confessasse que à primeira vista
vislumbrei nela um estranho encanto. Para definir esse encanto, não
saberia compará"-lo a nada melhor \linebreak

\quebra

\noindent{}que o das muito belas
mulheres adiantadas em anos, que no entanto já não envelhecem e cuja beleza
conserva a penetrante magia das ruínas. Seu aspecto era, a um só tempo,
imperioso e desalinhado, e seus olhos, embora abatidos, continham
uma força fascinante. O que mais me impressionou foi o mistério em sua voz,
na qual eu reencontrava os mais deliciosos \textit{contralti} e
também um pouco da rouquidão das gargantas lavadas sem cessar pela
aguardente.

``Você quer conhecer meu poder?'', disse a
falsa deusa com sua voz paradoxal e encantadora.
“Escute”.

E ela embocou então uma gigantesca trombeta ornada, feito flauta de
cana, com fitas trazendo as manchetes de todos os jornais do universo, e
por esta trombeta ela gritou meu nome, que rolou assim
espaço afora, com o ruído de cem mil trovões, e voltou a mim repercutido pelo
eco do mais longínquo planeta.

``Diabos!'' falei, meio subjugado, ``está aí algo precioso!'' Porém, ao examinar
mais atentamente a sedutora virago, tive a vaga impressão de
reconhecê"-la, por tê"-la visto brindando com alguns engraçadinhos
que conheço; e o som rouco do cobre me trouxe aos ouvidos não
sei que lembrança de uma trombeta prostituída.

De modo que respondi, com todo o meu desprezo: ``Vá embora! Não
nasci para desposar a amante de certos indivíduos que nem quero
nomear''.

De tão corajosa abnegação eu tinha decerto o direito de me orgulhar.
Infelizmente, porém, acordei e toda a minha força me abandonou.
``Na verdade,'' pensei, ``devia estar
dormindo um sono muito pesado para demonstrar tamanhos escrúpulos.'' Ah! quem dera
eles voltassem quando estou acordado, não me faria de tão
delicado!

E os invoquei em voz alta, suplicando que me perdoassem, me
oferecendo para desonrar"-me quantas vezes fosse preciso para merecer
seus favores. Mas eu os tinha, por certo, fortemente ofendido, pois que
nunca mais voltaram.

\quebra\section[O crepúsculo da tarde]{o crepúsculo da tarde}

A tarde cai. Faz-se um grande sossego nos pobres espíritos
cansados do labor da jornada, e seus pensamentos adotam agora as cores
suaves e indecisas do crepúsculo.

Chega porém, do alto da montanha à minha sacada, através das nuvens
transparentes do entardecer, um grande alarido composto por 
quantidade de gritos discordantes, que o espaço transforma numa lúgubre harmonia, qual a da maré subindo ou da tempestade despertando.

Quem são os desventurados que o entardecer não acalma e que, como as
corujas, tomam a vinda da noite por um sinal de sabá? Esta sinistra
ululação nos chega do negro hospício encarapitado na montanha; e
ao entardecer, fumando e contemplando o repouso do imenso vale eriçado
de casas em que cada janela diz: ``Aqui agora existe paz!
Aqui existe a alegria da família!'', eu posso, com o
soprar do vento lá de cima, embalar meu pensamento espantado nesta imitação
das harmonias do inferno.

O crepúsculo excita os loucos. Lembro que tive dois amigos a quem o
crepúsculo deixava bem doentes. Um deles desconhecia então qualquer
relação de amizade e educação e maltratava, feito um bruto,
quem lhe aparecia pela frente. Eu o vi atirar ao rosto de um
mordomo um frango excelente, no qual julgava enxergar não sei que
insultante hieróglifo. O entardecer, precursor das volúpias profundas,
estragava"-lhe as coisas mais suculentas.

\quebra

O outro, um ambicioso ferido, à medida que a luz sumia
ia se tornando mais azedo, mais sombrio, mais mesquinho. Ainda indulgente e sociável
durante o dia, era impiedoso à tardinha; e não era apenas nos outros,
como também em si mesmo, que ele exercia raivosamente sua 
crepusculosa mania.

O primeiro morreu louco, incapaz de reconhecer a mulher e o filho; o
segundo traz em si a inquietação de um perpétuo mal"-estar, e fosse
ele gratificado com todas as honrarias que repúblicas
e príncipes podem conferir, acho que o crepúsculo ainda acenderia dentro dele o ardente
desejo de distinções imaginárias. A noite, que no 
espírito deles punha trevas, no meu faz a luz; e, embora não seja raro ver a mesma
causa gerar dois efeitos contrários, isto sempre me deixa como 
intrigado e alarmado.

Oh, noite! Oh, trevas refrescantes! Vocês são para mim sinal de uma
festa interior, a redenção de uma angústia! Cintilação das
estrelas, explosão das lanternas, são na solidão das
planícies, nos pétreos labirintos de uma capital, o fogo de artifício da
deusa Liberdade!

Crepúsculo, que doce e terno é você! Os clarões rosados que ainda se
demoram no horizonte como a agonia do dia sob a opressão
vitoriosa de sua noite, os fogos dos candelabros, que criam manchas de
um vermelho opaco nas últimas glórias do poente, 
os pesados panejamentos que uma mão invisível atrai das profundezas do Oriente,
imitam todos os sentimentos complicados que lutam no coração do homem
nas horas solenes da vida.

Lembra igualmente essas estranhas vestes de bailarina, com uma
gaze transparente e escura deixando entrever o esplendor fosco de
uma saia deslumbrante, igual sob um negro presente transparece o delicioso
passado; e as estrelas vacilantes de ouro e prata com que ela foi
polvilhada representam esses fogos da fantasia que só se acendem direito
no luto profundo da noite.

\quebra\section[A solidão]{a solidão}

Diz-me um gazeteiro filantropo que a solidão é ruim para o homem; e em
apoio à sua tese cita, como todo incrédulo, palavras dos Padres
da Igreja.

Sei que o Demônio frequenta de bom grado os locais áridos, e que o
Espírito de assassinato e lubricidade se inflama maravilhosamente 
nas solidões. Mas é bem possível que esta solidão só seja um perigo
para a alma ociosa e divagante que a povoa com suas paixões e quimeras.

É certo que um tagarela, cujo supremo prazer consiste em falar do alto
de uma cátedra ou tribuna, estaria seriamente arriscado a
ficar louco de atar na ilha de Robinson.\protect\footnote{ Robinson 
Crusoé, personagem da obra de mesmo nome do escritor inglês Daniel Defoe (1660"-1731).} Não exijo
do meu gazeteiro as corajosas virtudes de Crusoé, mas peço que não
decrete culpados os amantes da solidão e do mistério.

Existem nas nossas raças tagarelas indivíduos capazes de aceitar o suplício supremo com menos
repulsa, caso lhes fosse permitido proferir copiosa arenga do
alto do cadafalso, sem receio que os tambores de
Santerre\protect\footnote{  Claude Santerre, general da Guarda Nacional durante a Revolução
Francesa, ordenou um rufar de tambores para impedir que fossem ouvidas
as últimas palavras de Luis \textsc{xvi}, proferidas do cadafalso.} lhes cortassem intempestivamente a palavra.

Não os lastimo, pois imagino que suas efusões oratórias lhes
propiciam volúpias iguais às que outros encontram no silêncio e
recolhimento; mas desprezo"-os.

Desejo, antes de tudo, que meu maldito gazeteiro me deixe divirtir-me a 
bel-prazer. “Então você nunca sente --- me diz ele num
tom nasal muito apostólico --- a necessidade

\quebra

\noindent{}de partilhar seus
prazeres?'' Vejam só que sutil invejoso! Sabendo que eu
menosprezo os seus, vem insinuar"-se nos meus, o hediondo
desmancha"-prazeres!

``A grande infelicidade de não poder estar
só!''\ldots\  diz La Bruyère em algum lugar,\protect\footnote{``Toda a 
nossa infelicidade advém do fato de não
podermos ficar sós.'' Jean de La Bruyére (1645"-1696),
\textit{``Les Caractères''}, capítulo De L'homme.} como para
envergonhar todos esses que correm se esquecer na
multidão, decerto temendo não suportarem a si mesmos.

``Quase todos os nossos males advêm de não termos sabido
ficar dentro de nosso quarto'', diz outro sábio,
Pascal,\protect\footnote{  Blaise Pascal (1623"-1662), \textit{Pensées}.}
acho, chamando assim de volta à cela do
recolhimento todos esses sobressaltados que buscam a felicidade 
no movimento e numa prostituição que eu poderia chamar de
\textit{fraternitária} se quisesse falar a bela língua de meu século.

\quebra\section[Os projetos]{os projetos}

Ele pensava, enquanto passeava num grande parque deserto:
``Que linda ela ficaria com um traje de corte, complicado
e fastuoso, descendo, na atmosfera de um belo entardecer, os
degraus de mármore de um palácio, diante dos grandes gramados e açudes!
Pois ela tem naturalmente um ar de princesa!''

Passando mais tarde numa rua, parou em frente a uma loja de gravuras
e, ao encontrar numa pasta uma estampa representando uma paisagem
tropical, pensou: ``Não! Não é num palácio que eu gostaria
de possuir sua querida vida. Não estaríamos \textit{em casa}. Aliás, essas
paredes crivadas de ouro não deixariam um só espaço para pendurar sua
imagem; nestas solenes galerias, não há um só recanto para a
intimidade. Realmente, \textit{aqui} é que teria de morar para cultivar
o sonho de minha vida''.

E enquanto analisava os detalhes da gravura, prosseguia mentalmente:
``À beira-mar, uma linda choupana de madeira, envolta
em todas essas árvores estranhas e reluzentes cujos nomes esqueci\ldots\ 
na atmosfera, um cheiro embriagante, indefinível\ldots\  na choupana, um
forte aroma de rosa e almíscar\ldots\  adiante, atrás da nossa
pequena propriedade, pedaços de mastros balançados pelo marulho\ldots\  à nossa volta, para além do quarto aclarado por uma luz rosada filtrada
pelos estores, decorada com esteiras novas e flores capitosas, raras poltronas de um rococó português, de madeira pesada e escura (em
que ela repousaria tão calma, tão arejada, tragando o fumo
levemente opiáceo!), para além da varanda, o tumulto dos pássaros bêbados de
luz e o tagarelar das negrinhas\ldots\  e à noite, para servir de
acompanhamento aos meus sonhos, o canto sentido das árvores musicais,
as melancólicas casuarinas! Sim, na verdade, é este o cenário que eu
buscava. Para que iria querer palácios?''

Mais adiante, ao percorrer uma larga avenida, avistou uma pousada
asseadinha onde, a uma janela enfeitada com cortinas de chita
estampada, se debruçavam dois rostos risonhos. E, logo:
``Meu pensamento'' --- refletiu --- ``deve ser muito vagamundo
para ir buscar tão longe o que está tão perto de mim. O prazer e a
felicidade estão na primeira pousada que aparece, na pousada do
acaso, tão fecunda em volúpias. Um bom fogo, faianças vistosas, um
jantar passável, um vinho rude e uma cama bem ampla com lençóis um
pouco ásperos, mas limpos; haverá coisa melhor?''

E ao voltar sozinho para casa, nesta hora em que os conselhos da
Sabedoria já não são sufocados pelo burburinho da vida exterior,
ele pensou: ``Tive hoje, em sonhos, três domicílios em que
encontrei igual prazer. Por que obrigar meu corpo a mudar de lugar, se minha alma é tão ligeira em viajar? E de que serve executar projetos,
se o projeto, em si, já é fruição suficiente?''

\quebra\section[A bela Dorothée]{a bela dorothée}

O sol oprime a cidade com sua luz direta e terrível; a areia está
ofuscante e o mar rebrilha. O mundo estupefato se rende, covarde e faz
a sesta, uma sesta que é como uma morte saborosa em que quem
dorme, semidesperto, desfruta as volúpias do aniquilamento.

Entretanto, Dorothée, altiva e intensa como o sol, avança pela rua
deserta, única alma viva a esta hora sob o azul imenso, criando sobre a
luz uma mancha brilhante e negra.

Ela avança, balançando brandamente o tronco tão fino sobre os
quadris tão amplos. Seu vestido de seda colante, de um tom claro e
rosado, sobressai intensamente nas trevas de sua pele e delineia com
precisão seu talhe longo, sua cintura arqueada e seu colo pontudo.

Sua sombrinha vermelha, peneirando a luz, projeta em seu rosto escuro a
maquiagem ensanguentada do seu reflexo.

O peso da enorme cabeleira quase azul puxa para trás sua cabeça
delicada, dando-lhe um porte triunfante e preguiçoso. Pesados pingentes
tilintam escondidamente em suas orelhas mimosas.

De quando em quando, a brisa de mar ergue a ponta de sua saia ondulante e
mostra a sua perna luzente e admirável; e seu pé, igual aos pés das deusas
de mármore que a Europa encerra em seus museus, imprime fielmente sua
forma na areia fina. Pois Dorothée é tão prodigiosamente coquete que o
prazer de ser admirada nela supera o orgulho da alforria e, mesmo sendo
livre, caminha descalça.

Ela avança assim, harmoniosamente, feliz de viver e sorrindo um branco
sorriso como se avistasse ao longe, no espaço, um espelho a refletir
seu andar e sua beleza.

Nesta hora em que mesmo os cães gemem de dor ao sol que os aferra,
que poderoso motivo leva a andar assim a preguiçosa Dorothée, bela e fria
como o bronze?

Por que terá deixado sua pequena choupana tão lindamente arrumada, onde
as flores e as esteiras compõem, a bem pouco custo, um perfeito
toucador; onde tem tanto prazer em se pentear, fumar, ser
abanada ou se  mirar no espelho dos seus grandes leques de plumas,
enquanto o mar, batendo na praia a cem passos dali, dá a seus devaneios
indecisos um poderoso e monótono acompanhamento, e que a marmita de
ferro, cozendo um ensopado de caranguejo com arroz e açafrão, lhe
manda do fundo do pátio seus excitantes aromas?

Ela talvez tenha um encontro com algum jovem oficial que em praias
distantes tenha ouvido falar, pelos seus companheiros, da famosa
Dorothée. Ela inevitavelmente há de suplicar, simples
criatura, que ele lhe descreva o baile da Ópera, e perguntará se
nele se pode ir de pés descalços, como nas danças de domingo em
que até as velhas cafrinas ficam bêbadas e loucas de alegria;
perguntará ainda se as belas damas de Paris são todas mais belas que
ela.

\quebra

Dorothée é admirada e mimada por todos, e seria completamente feliz se
não fosse obrigada a juntar piastra por piastra para resgatar sua
irmãzinha, que deve ter seus onze anos e já está madura, e tão
linda! Há de conseguir, sem dúvida, a boa Dorothée; o dono da criança é
tão sovina, demasiado sovina para entender outra beleza que não a das
moedas!

\quebra\section[Os olhos dos pobres]{os olhos dos pobres}

Ah! Você quer saber por que hoje a odeio. Será, sem dúvida, menos fácil
para você entender do que, para mim, explicar; pois você é, me
parece, o mais belo exemplo de impermeabilidade feminina que se possa
encontrar.

Tínhamos passado juntos um longo dia que me parecera curto. Tínhamos
deveras prometido um ao outro que todos os nossos pensamentos seriam
comuns e que nossas duas almas seriam de ora em diante uma só ---
um sonho que nada tem de original, afinal, se não o fato de, sonhado 
por todos os homens, não ter sido realizado por nenhum.

À noite, um pouco cansada, você quis sentar"-se frente a um café
novo, que formava a esquina com uma avenida nova, ainda apinhada de
cascalhos e já exibindo gloriosamente seus esplendores inacabados. O
café reluzia. Até o gás ostentava ali todo o ardor de um começo e
iluminava com toda força as paredes ofuscantes de brancura,
a superfície deslumbrante dos espelhos, o ouro das molduras e 
cornijas, os pajens de faces roliças puxados por cães de coleira, as
senhoras rindo para o falcão empoleirado em seus punhos, as ninfas e
deusas carregando na cabeça frutas, caças e patês, as Hebes e os
Ganimedes\protect\footnote{ Na mitologia grega, Hebe é a deusa da juventude e Ganimedes, um
príncipe troiano de quem Zeus fez o copeiro dos deuses.} 
estendendo os braços para oferecer a
pequena ânfora de licores ou o obelisco bicolor dos sorvetes variados;
toda a história e toda a mitologia a serviço da glutonaria.

\quebra

Bem em frente de nós, na calçada, quedava-se um bom homem de uns
quarenta anos, de rosto cansado, barba grisalha, com uma
das mãos segurando um menino e levando, no outro braço, uma criaturinha frágil demais
para andar. Ele fazia as vezes de babá e trouxera os filhos para
tomar o ar da noite. Todos em andrajos. Os três rostos eram
extraordinariamente sérios e os seis olhos contemplavam fixamente o café
novo com admiração igual, porém distintamente matizada pela idade.

Os olhos do pai diziam: ``Que bonito! Que bonito!
Parece que todo o ouro do pobre mundo veio encerrar"-se nessas
paredes''. Os olhos do menino: ``Que
bonito! Que bonito! Só que é uma casa onde só entra gente que não
é como a gente''. Quanto aos olhos do menorzinho, estavam
fascinados demais para expressar algo além de uma alegria estúpida e
profunda.

Dizem os cantadores que o prazer torna a alma boa e amolece o coração. Quanto a mim, naquela noite, a canção estava certa. Eu
me sentia não só comovido com aquela família de olhos, como envergonhado
com nossos copos e jarras, maiores que a nossa sede. Voltei meu
olhar para o seu, amor querido, para nele ler meu pensamento;
mergulhava nos seus olhos tão lindos e estranhamente doces, 
seus olhos verdes habitados pelo Capricho e inspirados pela Lua, e então
você disse: ``Não suporto essa gente, esses
olhos arregalados! Você não poderia pedir ao dono do café que os
afastasse daqui?''

\quebra

Tão difícil é se entender, meu anjo querido, e tão incomunicável é o
pensamento, mesmo entre quem se ama!

\quebra\section[Uma morte heroica]{uma morte heroica}

Fancioulle era um admirável bufão, e quase um amigo do Príncipe. Mas,
para as pessoas votadas ao cômico por profissão, as coisas sérias possuem
fatais atrativos e, mesmo que possa parecer estranho as ideias de
pátria e liberdade se apossarem despoticamente do cérebro de um
histrião, Fancioulle se envolveu certo dia numa conspiração formada por alguns
fidalgos descontentes.

Existem em todo lugar homens de bem para denunciar ao poder esses
indivíduos de humor atrabiliário que querem depor os príncipes e
efetuar, sem consultá"-la, a mudança de uma sociedade. Os senhores em
questão foram presos, assim como Fancioulle, e condenados a uma morte
certa.

Quero acreditar que o Príncipe ficou quase zangado ao
perceber seu ator favorito entre os rebeldes. O Príncipe não era
melhor nem pior que qualquer outro; mas uma excessiva sensibilidade o tornava,
em muitos casos, mais déspota e mais cruel que todos os seus pares.
Amante apaixonado das artes, excelente conhecedor por sinal, era de fato
insaciável em volúpias. Um tanto indiferente em relação aos homens e à
moral, verdadeiro artista ele próprio, como inimigo perigoso só conhecia
o Tédio, e os esquisitos esforços que fazia para evitar ou
vencer este tirano do mundo teriam decerto lhe valido, por parte de
um historiador severo, o epíteto de
monstro, caso fosse permitido escrever, nos seus
domínios, o que quer que não tendesse unicamente
para o prazer, e para o espanto, que é uma das mais delicadas formas de
prazer. A grande má sorte deste Príncipe foi nunca ter possuído
um teatro vasto o bastante para a sua genialidade. Há jovens Neros que
sufocam em limites demasiado estreitos, e cujo nome e boa vontade sempre serão ignorados nos séculos por vir. A este, a imprevidente
Providência concedera talentos maiores que seus Estados.

Correu de repente o boato de que o soberano pretendia indultar todos os
conjurados; e a origem deste boato foi o anúncio de um grande
espetáculo em que Fancioulle representaria um dos seus principais e
melhores papéis, e ao qual assistiriam inclusive, ao que diziam, os
cavalheiros condenados; sinal evidente, acrescentavam os espíritos
superficiais, das tendências generosas do Príncipe ofendido.

Vindo de um homem tão natural e voluntariamente excêntrico, tudo era
possível, até mesmo a virtude, até mesmo a clemência, sobretudo se ele nela 
contasse encontrar prazeres inesperados. Mas para quem,
como eu, lograva penetrar mais além nas profundezas daquela
alma curiosa e doentia, era muitíssimo mais provável que o Príncipe
quisesse avaliar os talentos cênicos de um homem condenado à
morte. Queria aproveitar a oportunidade para fazer uma experiência
fisiológica de interesse \textit{capital}, e conferir até que ponto as
faculdades habituais de um artista podiam ser mudadas ou alteradas
pela situação extraordinária em que ele se encontrava;
existiria, além disto, em sua alma, alguma intenção mais ou menos definida
de clemência? É um ponto que nunca se pôde esclarecer.

\quebra

Chegado afinal o grande dia, a pequena corte exibiu toda a sua pompa,
e seria difícil conceber, a menos de ter visto, tudo o que a classe
privilegiada de um pequeno Estado de recursos restritos é capaz de mostrar
numa verdadeira solenidade em matéria de esplendor. Aquela era duplamente
verdadeira, primeiro pela magia do luxo ostentado, depois pelo
interesse moral e misterioso a ela ligado.

O senhor Fancioulle se destacava sobretudo nos papéis mudos ou de
pouca fala que, não raro, são os principais nesses dramas feéricos
cujo objeto é representar simbolicamente o mistério da vida. Ele entrou
em cena com leveza e uma naturalidade perfeita, o que contribuiu
para fortalecer, no nobre público, a ideia de doçura e perdão.

Quando se diz de um ator: “É um bom ator'',
está"-se usando uma fórmula que implica que por detrás do personagem
ainda transparece o ator, isto é, o esforço, a vontade. Ora, se um ator
conseguisse ser, em relação ao personagem que lhe cabe expressar,
o que seriam as melhores estátuas da antiguidade, milagrosamente
animadas, vivas, andantes, videntes, em relação à ideia usual e
confusa de beleza, este seria, sem dúvida, um caso singular e
inteiramente imprevisto. Fancioulle foi, naquela noite, uma perfeita
idealização, que era impossível não supor viva, possível, real. O bufão
ia, vinha, ria, chorava, se convulsionava, com uma auréola indestrutível
ao redor da cabeça, auréola invisível para todos, porém visível para mim,
e na qual se mesclavam, num amálgama estranho, os clarões da Arte e a
glória do Martírio. Fancioulle introduzia, por não sei que graça
especial, o divino e o sobrenatural até nos mais extravagantes
gracejos. Minha pena estremece, e lágrimas de uma emoção sempre
presente me vêm aos olhos enquanto procuro descrever esta noite
inesquecível. Fancioulle estava me provando de maneira peremptória,
irrefutável, que a embriaguez da Arte é mais apta que qualquer
outra a ocultar os terrores do abismo; que o gênio pode representar
à beira do túmulo, com uma alegria que o impede de ver o túmulo,
perdido que está num paraíso que exclui toda ideia de túmulo e destruição.

Todo aquele público, por enfastiado e frívolo que pudesse ser, logo
se submeteu ao todo"-poderoso domínio do artista. Ninguém mais
sonhava com morte, luto, ou suplícios. Cada qual se entregou,
sem receio, às volúpias multiplicadas oferecidas pela visão de uma obra-prima 
da arte viva. As explosões de alegria e admiração abalaram
repetidas vezes os arcos do edifício com a energia de um trovão
continuado. O próprio Príncipe, embevecido, juntou seu aplauso aos da
corte.

Entretanto, para um olhar perspicaz, sua embriaguez não era isenta de
imisção. Estaria se sentido derrotado em seu poder de déspota? Humilhado em
sua arte de aterrorizar os corações e embotar os espíritos? Frustrado
em suas esperanças e escarnecido em suas previsões? Tais suposições,
não exatamente justificadas, mas não de todo injustificáveis,
atravessaram-me a mente enquanto eu contemplava o rosto do príncipe,
no qual nova palidez se vinha incessantemente juntar-se à sua palidez habitual, como
a neve se junta à neve. Seus lábios se apertavam mais e mais, e seus
olhos se iluminavam com um fogo interior semelhante ao da inveja e do
rancor, mesmo enquanto aplaudia ostensivamente os talentos do seu velho
amigo, o estranho bufão, que bufoneava tão bem a morte. Num certo
momento, vi sua Alteza se virar para um pequeno pajem postado atrás
dela, e lhe falar ao ouvido. A fisionomia marota do menino se iluminou
num sorriso; e então ele deixou rapidamente o camarote do príncipe como
que para cumprir alguma ordem urgente.

Alguns minutos mais tarde, um som de apito, agudo, prolongado, interrompeu
Fancioulle num de seus melhores momentos, e lacerou a um só tempo
ouvidos e corações. E do lugar da sala de onde surgira essa
censura inesperada, uma criança se precipitava num corredor, com
risos contidos.

Fancioulle, abalado, desperto de seu sonho, primeiro fechou os olhos,
quase em seguida os reabriu, excessivamente dilatados, então abriu a
boca como que a respirar convulsivamente, titubeou um pouco para
frente, um pouco para trás, e então caiu morto sobre o palco.

O apito, rápido feito um gládio, teria realmente frustrado o carrasco?
Teria o próprio Príncipe adivinhado toda a homicida eficácia do seu
ardil? É permitido duvidar. Terá lastimado seu caro e inimitável
Fancioulle? É doce e legítimo pensá"-lo.

Os fidalgos culpados haviam desfrutado pela última vez do espetáculo
teatral. Na mesma noite foram riscados da vida.

Desde então, vários mímicos, devidamente apreciados em diferentes
países, vieram representar perante a corte de ***; nenhum deles, porém,
conseguiu evocar os fabulosos talentos de Fancioulle, nem alcançar
os mesmos \textit{favores}. 

\quebra\section[A moeda falsa]{a moeda falsa}

Enquanto nos afastávamos da tabacaria, meu amigo fez uma cuidadosa
seleção do seu dinheiro; no bolso esquerdo do colete enfiou
moedinhas de ouro, no direito moedinhas de prata; no bolso esquerdo da calça, um monte de tostões, e no direito, enfim, uma moeda de prata
de dois francos que ele tinha especialmente examinado.

``Minuciosa e singular distribuição!'', pensei
comigo mesmo.

Deparamos com um pobre que, tremendo, nos estendeu seu boné. Não
conheço nada tão inquietante como a eloquência muda desses olhos
suplicantes que contêm, ao mesmo tempo, tanta humildade e tanta censura para o homem sensível que neles
sabe ler. Ele ali encontra algo que se
aproxima da profundidade de sentimentos complicados nos olhos
lacrimejantes dos cães açoitados.

A oferta do meu amigo foi bem mais considerável que a minha, e eu lhe
disse: ``Você tem razão; depois do prazer de
surpreender"-se, não há prazer maior que causar uma surpresa''. 
``Era a moeda falsa'', respondeu tranquilamente, como justificando sua prodigalidade.

Mas, no meu miserável cérebro sempre ocupado em complicar as coisas (com
que cansativa faculdade a natureza me brindou!),
surgiu de repente a ideia de que tal atitude, da parte do meu amigo, só
era desculpável pelo desejo de criar um acontecimento na vida do
pobre diabo, talvez até de \linebreak

\quebra

\noindent{}conhecer as consequências diversas, funestas
ou outras, que uma moeda falsa é capaz de gerar na mão de um mendigo. Não
poderia multiplicar"-se em moedas verdadeiras? Ou então, não poderia
levá"-lo à prisão? Um taberneiro, um padeiro, por exemplo, talvez o
mandasse prender como falsário ou propagador de dinheiro falso. Do
mesmo modo, a moeda falsa talvez fosse, para um pobre especuladorzinho,
o germe de uma riqueza de alguns dias. E assim prosseguia minha
fantasia, dando asas ao espírito do meu amigo e tirando todas as
deduções possíveis de todas as hipóteses possíveis.

Ele, porém, bruscamente interrompeu meu devaneio, retomando minhas
próprias palavras: “Sim, você tem razão; não há prazer
mais doce que o de surpreender um homem dando"-lhe mais do que ele
espera''.

Olhei no fundo dos seus olhos, e me apavorei ao ver que
brilhavam com incontestável candura. Então vi claramente que ele
quisera, a um só tempo, fazer caridade e um bom negócio; ganhar quarenta
tostões e o coração de Deus; ganhar o paraíso com economia; enfim,
obter de graça um atestado de homem caridoso. Eu quase teria lhe
perdoado o desejo do gozo criminoso de que há pouco o supunha capaz;
teria achado curioso, singular, que se divertisse comprometendo
os pobres; mas jamais lhe perdoarei a inépcia do seu cálculo. Nunca é
perdoável sermos maus, mas há certo mérito em \linebreak

\quebra

\noindent{}saber que o somos; e o
mais irreparável dos vícios é praticar o mal por tolice.

\quebra\section[O jogador generoso]{o jogador generoso}

Ontem, em meio à multidão da avenida, senti que era tocado de leve por
um Ser misterioso que eu sempre desejara conhecer, e que reconheci de
imediato mesmo sem nunca tê"-lo visto. Havia decerto nele, em
relação a mim, um desejo análogo, pois lançou"-me, ao passar uma
piscadela significativa à qual me apressei em obedecer. Segui"-o
atentamente, e logo estava descendo atrás dele numa morada subterrânea,
deslumbrante, em que resplandecia um luxo do qual nenhuma das
habitações superiores de Paris poderia dar um exemplo aproximado.
Pareceu"-me singular que eu pudesse ter passado tantas vezes ao lado
daquele prestigioso esconderijo sem perceber"-lhe a entrada. Ali
reinava uma atmosfera deleitável, embora capitosa, que levava a
esquecer quase instantaneamente todos os fastidiosos horrores da
vida; ali se respirava uma beatitude sombria, análoga à que devem ter
experimentado os comedores de lótus\protect\footnote{  A história 
dos lotófagos é contada por Homero no Canto \textsc{ix} da
\textit{Odisseia}.} quando, ao
desembarcarem numa ilha encantada, iluminada pelos clarões de uma tarde
eterna, sentiram nascer em si, ao som entorpecente das melodiosas
cascatas, o desejo de não mais rever seus lares, suas mulheres, seus
filhos, e não mais retornar às altas vagas do mar.

Havia ali rostos estranhos de homens e mulheres marcados por uma beleza
fatal, que me parecia já ter avistado em épocas e terras que era
impossível lembrar exatamente, e me inspiravam antes uma simpatia
fraternal do que esse temor que comumente nasce ante o
desconhecido. Se eu quisesse tentar definir de algum modo a
singular expressão do seu olhar, diria que nunca vi olhos
brilhando com mais energia de horror ao tédio e desejo imortal de
se sentir vivo.

Meu anfitrião e eu já éramos, ao nos sentarmos, perfeitos e velhos 
amigos. Comemos, bebemos além da medida toda sorte de vinhos
extraordinários e, fato não menos extraordinário, pareceu-me que passadas várias horas estávamos ambos igualmente sóbrios. Entretanto,
o jogo, este prazer sobre"-humano, tinha repartido em vários
intervalos nossas frequentes libações, e devo dizer que eu havia
apostado e perdido minha alma, num mútuo acordo, com indiferença e
leviandade heroicas. A alma é uma coisa tão impalpável, tantas vezes
inútil e às vezes tão incômoda, que eu apenas sentia, em relação a esta
perda, pouco menos emoção que se tivesse extraviado, durante um
passeio, meu cartão de visitas.

Fumamos demoradamente alguns charutos, cujo sabor e aroma incomparáveis
traziam à alma a nostalgia de terras e venturas ignoradas; e,
bêbado dessas tantas delícias, ousei, num acesso de familiaridade que
não pareceu desagradá"-lhe, exclamar, apossando"-me de uma taça cheia
até à borda: “A sua imortal saúde, velho
Bode!''.

Conversamos também sobre o universo, sua criação e futura destruição;
sobre a grande ideia do século, ou seja, sobre o progresso e a
perfectibilidade e, genericamente, sobre todas as formas da enfatuação
humana. Sobre este assunto, Sua Alteza não esgotava brincadeiras leves
e irrefutáveis, e se expressava com uma suavidade na dicção e uma
tranqui-

\quebra

\noindent{}lidade no gracejo que não encontrei em nenhum dos mais famosos
conversadores da humanidade. Ele me explicou o absurdo das diferentes
filosofias que tinham até o momento tomado conta do cérebro humano e
até dignou confidenciar"-me alguns princípios fundamentais cujos
benefícios e propriedade não me convém partilhar com quem quer que seja.
Não se queixou de maneira nenhuma da má reputação de que goza em todas
as partes do mundo, assegurou"-me de que ela mesma era a pessoa mais
interessada na destruição da \textit{superstição}, e confessou que uma só vez
tinha sentido medo, em relação ao seu próprio poder, e isso no dia em
que ouvira um pregador, mais sutil que seus colegas, exclamar da
cátedra: “Caros irmãos, nunca se esqueçam, quando
ouvirem louvar o progresso das luzes, que a maior astúcia do diabo é
persuadi"-los de que ele não existe!''.

A lembrança deste célebre orador nos conduziu naturalmente ao assunto
das academias, e meu estranho conviva me afirmou que não se negava, em
muitos casos, a inspirar a pena, a palavra e a consciência dos
pedagogos e que quase sempre assistia pessoalmente, se bem que invisível,
a todas as sessões acadêmicas.

Encorajado por tantas bondades, perguntei-lhe se tinha notícias de
Deus, se o tinha avistado recentemente. Respondeu"-me com uma
despreocupação matizada de certa tristeza: “Nos
cumprimentamos quando nos encontramos, mas como dois velhos
cavalheiros em que uma cortesia inata \linebreak

\quebra

\noindent{}não chega a apagar
por completo a lembrança de antigos rancores''.

“É pouco provável que Sua Alteza tenha alguma vez dado tão longa
audiência a um simples mortal, e eu temia abusar. Por fim, quando a
aurora, estremecendo, já embranquecia as vidraças, o célebre
personagem, decantado por tantos poetas e servido por tantos filósofos
que, sem saber, trabalham para a sua glória, disse"-me: Quero
que guarde de mim uma boa recordação, e provar que Eu, de
quem se diz tanto mal, sou por vezes um \textit{bom diabo}, para usar uma de
suas locuções corriqueiras. A fim de o compensar pela perda
irremediável de sua alma, dou"-lhe o cacife que teria ganho se a
sorte o tivesse acompanhado, ou seja, a possibilidade de aliviar e 
vencer, durante toda a sua vida, esta estranha afecção que é o Tédio,
fonte de todas as suas doenças e todos os seus miseráveis progressos.
Jamais haverá desejo por você formulado que eu não o ajude a realizar;
você reinará sobre seus vulgos semelhantes; será cumulado de
adulações, e até adorações; a prata, o ouro, os diamantes, os
palácios feéricos virão buscá"-lo e suplicar que os aceite, sem
você ter feito um esforço sequer para obtê"-los; você mudará de
pátria e região tantas vezes quanto a sua fantasia ordenar; se
embriagará de volúpias, sem lassidão, em terras encantadoras onde faz
sempre calor e as mulheres têm o cheiro bom das flores
--- et cetera\ldots\  et cetera\ldots\ '' acrescentou, erguendo"-se e
me despedindo com um belo sorriso.

\quebra

Não fosse o temor de me humilhar perante tamanha assembleia, eu teria
de bom grado caído aos pés daquele jogador generoso para agradecer
sua incrível munificência. Mas pouco a pouco, após tê"-lo deixado, a
incurável desconfiança foi penetrando meu peito; já não ousava acreditar em
tão prodigiosa sorte e, ao deitar"-me, ainda rezando por um
resquício de hábito inepto, repetia numa semi"-sonolência:
“Meu Deus! Senhor, meu Deus! Faça com que o diabo cumpra
com sua palavra!''

\quebra\section[A corda]{a corda}
\begin{flushright}
\textit{Para Edouard Manet}
\end{flushright}

“As ilusões'' --- me dizia meu amigo --- “talvez sejam
tantas quanto as relações dos homens entre si, ou dos homens com as
coisas. E quando a ilusão desaparece, ou seja, quando enxergamos o ser
ou o fato tal como existe fora de nós, experimentamos um sentimento
estranho, complicado em parte pela perda do fantasma esvanecido, em
parte pela agradável surpresa ante a novidade, ante o fato real.
Se existe um fenômeno evidente, trivial, sempre igual, e de natureza
tal que é impossível se enganar, é o amor materno. É tão difícil supor
uma mãe sem amor materno quanto uma luz sem calor; não será
perfeitamente legítimo atribuir ao amor materno todas as ações e
palavras de uma mãe em relação ao seu filho? E, no entanto, escute esta
pequena história em que fui singularmente mistificado pela ilusão mais
natural''.

“Minha profissão de pintor me leva a olhar atentamente os
rostos, as fisionomias que se apresentam em meu caminho, e você sabe que
prazer nos traz esta faculdade, que torna aos nossos olhos a vida mais
viva e mais significativa que para os outros homens. No bairro afastado
onde moro, em que amplos espaços gramados ainda separam os prédios,
observei frequentemente um menino cuja fisionomia ardente e marota,
mais que todas as outras, me seduziu a princípio. Ele posou mais de
uma vez para mim, e eu o \linebreak

\quebra

\noindent{}transformei ora em pequeno cigano, ora em
anjo, ora em Amor mitológico. Eu o fiz usar o violino do andarilho, a
Coroa de Espinhos, e os Pregos da Paixão, e a Tocha de Eros. Enfim,
aquela graça toda do garoto causava-me tamanho prazer que 
um dia roguei aos seus pais, gente pobre, que consentissem em me cedê"-lo,
prometendo bem vesti"-lo, dar"-lhe algum dinheiro e não lhe impor 
outra tarefa que não limpar meus pincéis e fazer minhas compras. O
menino, depois de limpo, ficou encantador, e a vida que levava
comigo lhe parecia um paraíso se comparada à que suportaria no casebre
paterno. Devo dizer, porém, que o homenzinho por vezes me espantou com
singulares crises de tristeza precoce, e logo manifestou um gosto
imoderado pelo açúcar e pelos licores; de sorte que um dia, ao constatar que apesar de minhas inúmeras advertências ele cometera
um novo furto desta espécie, ameacei mandá"-lo de volta para seus pais.
Em seguida saí, e meus negócios me retiveram bastante tempo fora de
casa.

“Qual não foi meu horror e meu espanto quando, ao chegar em
casa, o primeiro objeto que atraiu meu olhar foi meu homenzinho, o
maroto companheiro de minha vida, enforcado na porta deste
armário!\protect\footnote{  O pintor Edouard Manet (ver nota 1. do poema 
\textsc{xiii}) tinha nesta época
por modelo um garoto chamado Alexandre, que inspirou muitas de suas
pinturas (\textit{O menino e o cão}, \textit{O guitarrista}). Alexandre, de fato,
foi achado enforcado no ateliê de Manet em 1861. Nada comprova,
entretanto, a atitude da mãe relatada por Baudelaire no poema.} 
Seus pés quase encostavam no assoalho; uma
cadeira, que ele decerto empurrara com os pés, estava derrubada ao
seu lado; sua cabeça estava convulsivamente inclinada sobre um dos 
ombros; seu rosto, intumescido, e seus olhos, arregalados com 
assustadora fixidez, primeiro me deram ilusão de vida.
Desenforcá"-lo não

\quebra

\noindent{}era tarefa tão fácil como você poderia
pensar. Ele já estava bastante rijo, e eu sentia inexplicável repugnância
em deixá"-lo bruscamente cair ao chão. Precisei
sustê"-lo por inteiro com um dos braços e, com a mão do outro braço,
cortar a corda. Mas, feito isto, não estava tudo pronto; o monstrinho
usara um cordão bem fino que penetrara profundamente nas
carnes, e eu tinha agora, com uma tesoura fina, de procurar a corda 
entre duas dobras da inchação para livrar seu pescoço.

“Omiti lhe contar que eu rapidamente chamara por
socorro; mas todos os vizinhos se negaram a vir me ajudar,
fiéis neste sentido aos hábitos do homem civilizado, que nunca quer,
não sei por quê, se envolver em histórias de enforcado. Chegou enfim um
médico, o qual declarou que a criança estava morta há várias horas.
Quando, mais tarde, tivemos de despi"-la para o sepultamento, a
rigidez cadavérica era tal que, desistindo de flexionar seus membros,
tivemos de lacerar e cortar suas roupas para tirá"-las.

“O delegado a quem tive, naturalmente, de declarar o
acidente, me olhou atravessado e disse: ``Isto tudo é meio
suspeito'', movido, sem dúvida, por um desejo inveterado e
por um hábito profissional de assustar, por via das dúvidas, tanto os
inocentes quanto os culpados.

“Faltava cumprir uma tarefa suprema que me causava, só
de pensar, uma angústia terrível: precisava avisar os pais. Meus pés
se negavam a me levar. Tive, enfim, esta coragem.

\quebra

\noindent{}Mas, para o meu grande
espanto, a mãe se manteve impassível, não gotejou nenhuma lágrima do canto
dos seus olhos. Atribuí esta estranheza ao próprio horror que devia estar
sentindo e lembrei da máxima conhecida: ``As dores mais
terríveis são as dores caladas''. Quanto ao pai,
contentou"-se em dizer, de um jeito meio entorpecido, meio pensativo:
“Afinal, talvez fosse melhor assim; ele, de qualquer
forma, teria acabado mal!''

“Entretanto, o corpo estava estendido em meu sofá; assistido 
por uma empregada, eu tratava dos últimos preparativos, quando
entrou a mãe no meu ateliê. Queria ver, dizia, o cadáver do seu
filho. Eu não podia, na verdade, impedi"-la de embriagar"-se com sua
desgraça e negar"-lhe aquele supremo e sombrio consolo. Em seguida,
rogou"-me que lhe mostrasse o local onde seu menino se
enforcara. “Oh! não!'' --- respondi ---, “vai ser doído
para a senhora''. E, ao voltar involuntariamente os olhos
para o armário funesto, percebi, com um nojo mesclado de horror e
raiva, que o prego tinha ficado cravado na madeira, com um comprido
pedaço de corda ainda pendurado. Adiantei"-me rapidamente para arrancar
aqueles últimos vestígios da desgraça e estava para jogá"-los 
pela janela aberta, quando a pobre mulher agarrou meu braço e disse com voz
irresistível: “Oh! Senhor! Deixe isto para mim! Eu lhe peço! Eu
lhe suplico!” Seu desespero, assim me pareceu, decerto a
perturbara tanto que ela agora se tomava de ternura pelo que 
servira de instrumento para

\quebra

\noindent{}a morte do filho, e queria guardá"-lo
como a uma horrível e cara relíquia. E apoderou"-se do prego e
do cordão''.

“Enfim! enfim! estava tudo terminado. Só me restava voltar
ao trabalho, ainda mais intensamente que de costume, para ir expulsando
aos poucos o pequeno cadáver que assombrava as dobras do meu cérebro
e cujo fantasma me cansava com seus grandes olhos fixos. Dia
seguinte, porém, recebi um pacote de cartas: umas dos inquilinos do meu prédio,
algumas dos prédios vizinhos; uma do primeiro andar, outra do
segundo, outra do terceiro, e assim por diante; umas em estilo
semibrincalhão, como tentando disfarçar com um aparente gracejo a
sinceridade do pedido, outras pesadamente atrevidas e sem ortografia,
mas todas tendendo ao mesmo objetivo, ou seja, obter de mim um pedaço
da funesta e beatífica corda.'' Entre os signatários havia, devo dizer, mais mulheres que homens,
mas nem todos, acredite, pertenciam à classe ínfima e vulgar.
Guardei aquelas cartas.

“E então, de súbito, fez-se uma luz no meu cérebro, e
compreendi por que a mãe fazia tanta questão de me arrancar o cordão e
através de que comércio pretendia consolar"-se.''

\quebra\section[As vocações]{as vocações}

Num belo jardim em que os raios de um sol outonal pareciam se demorar à
vontade, sob um céu já esverdeado em que nuvens de ouro flutuavam feito
continentes em viagem, quatro lindas crianças, quatro meninos, sem
dúvida cansados de brincar, conversavam entre si.

Dizia um deles: “Ontem, levaram-me ao teatro. Em palácios
grandes e tristes, por trás dos quais se vê o mar e o céu, homens e
mulheres, sérios e tristes também, mas bem mais bonitos e mais bem 
vestidos que aqueles que vemos em todo lugar, falam com voz
cantante. Eles se ameaçam, suplicam, se desolam, e volta e meia apoiam
a mão num punhal enfiado em seu cinto. Ah! É tão bonito!
As mulheres são bem mais lindas e mais altas que as que
vêm nos visitar e embora, com seus grandes olhos fundos e suas faces
inflamadas, tenham um ar terrível, não podemos deixar de amá"-las.
Sentimos medo, vontade de chorar e, no entanto, estamos contentes\ldots\  E,
além disso, o que é mais singular, dá vontade de estar vestido da mesma
maneira, de dizer e fazer as mesmas coisas e falar com a mesma
voz\ldots\ ”

Uma das quatro crianças, que há alguns segundos já não escutava o
discurso do seu companheiro e observava com fixidez surpreendente não
sei que ponto no céu, disse de repente: “Olhem, olhem
lá\ldots\ ! Vocês \textit{o} estão vendo? Está sentado naquela nuvenzinha isolada,
naquela nuvenzinha cor de fogo, que anda de mansinho. \textit{Ele} também
parece estar nos olhando''.

\quebra

“Ora, mas quem?'' perguntaram os outros.

“Deus!'' ele respondeu, com um perfeito tom de
convicção. Ah! Já está bem longe; daqui a pouco
já não vão mais conseguir vê"-lo. Na certa está viajando, a
visitar todos os países. “Vejam, vai passar por trás daquela fileira
de árvores quase lá no horizonte\ldots\  e agora está descendo por
trás do campanário\ldots\  Ah! Já não se vê mais!'' E
o menino ficou muito tempo voltado para aquele lado, fixando, na linha
que separa a terra do céu, uns olhos em que brilhava uma inexprimível
expressão de êxtase e saudade.

“Como é bobo, esse aí, com esse Deus que só ele consegue
enxergar!'' disse então o terceiro, cuja pessoinha inteira
se destacava por uma vivacidade e vitalidade singulares.
“E eu, vou contar como aconteceu comigo algo que
nunca aconteceu com vocês, e que é um pouco mais interessante que esse seu
teatro e suas nuvens. --- Dias atrás, meus pais me levaram com
eles numa viagem, e como no albergue em que paramos não havia camas
suficientes para nós todos, ficou decidido que eu dormiria na mesma
cama que minha empregada.” Ele atraiu seus companheiros
para mais perto de si e falou com voz mais baixa. --- “É
uma singular sensação, sabe, a de não estar deitado sozinho e de estar
numa cama com a empregada, no escuro. Como eu não dormia, fiquei
brincando, enquanto ela dormia, de passar minha mão nos seus braços, 
seu pescoço, seus ombros. Os braços e o pescoço dela são bem
maiores que os de todas as outras mulheres, e sua pele é tão macia, tão
macia, que lembra papel de cartas ou papel de seda. Eu sentia tanto
prazer que teria continuado muito tempo assim se não tivesse tido
medo: medo, primeiro, de acordá"-la, e depois, medo de não sei quê.
Então enfiei minha cabeça nos seus cabelos, que lhe caíam nas costas
espessos feito uma crina, e cheiravam tão bem, garanto, como as
flores do jardim a esta hora. Quando puderem, tentem fazer o mesmo, e vão ver!''

O jovem autor desta prodigiosa revelação tinha, ao fazer seu relato, os
olhos arregalados numa espécie de estupefação do que ele ainda
sentia, e os raios do sol poente, deslizando pelos cachos ruivos da sua
cabeleira desgrenhada, nela acendiam como uma auréola sulfurosa de
paixão. Era fácil perceber que aquele ali não desperdiçaria a vida
procurando a Divindade nas nuvens, e que seguidamente a encontraria 
em outro lugar.

Por fim, disse o quarto: “Vocês sabem que não me divirto
nada lá em casa; nunca me levam a um espetáculo; meu tutor é avarento
demais; Deus não liga para mim e para o meu tédio, e não tenho uma
empregada bonita para me mimar. Muitas vezes tive a impressão que meu prazer
seria andar sempre em frente, sem saber para onde, sem que ninguém se
preocupasse, e ver sempre novas terras. Nunca estou bem em lugar
nenhum, e sempre acho que estaria melhor onde não estou. Pois bem! Eu
vi, na última feira da vila vizinha, três homens que vivem como eu
gostaria de viver. Vocês, não prestaram atenção neles. Eram
altos, quase negros e muito altivos, mesmo que em andrajos, com jeito
de não precisarem de ninguém. Seus grandes olhos \linebreak

\quebra

\noindent{}escuros ficaram
absolutamente brilhantes quando tocaram música; uma música tão
surpreendente que dá vontade ora de dançar, ora de chorar, ou as duas
coisas ao mesmo tempo, e enlouqueceríamos se os escutássemos por muito
tempo. Um deles, arrastando o arco no violino, parecia contar uma
aflição; e o outro, fazendo saltitar seu martelinho nas cordas de um
pianinho pendurado em seu pescoço com uma correia, parecia zombar
do lamento do seu companheiro, enquanto o terceiro, de quando
em quando, batia seus címbalos com uma violência extraordinária. Tão
satisfeitos estavam consigo mesmos que seguiram tocando sua música de
selvagens mesmo depois que a multidão se dispersou. Por fim, jjuntaram
seus tostões, puseram a bagagem nas costas e foram-se embora. Eu,
querendo saber onde moravam, os segui de longe, até a orla da
floresta, onde só então entendi que não moravam em lugar nenhum.

Então um deles disse: “Será preciso montar a
barraca?''.

“Não, ora essa!'' respondeu o outro,
“está uma noite tão linda!''.

O terceiro dizia, contando a receita: “Essa gente não
sente a música, e suas mulheres dançam feito ursos. Felizmente, em menos
de um mês estaremos na Áustria, onde encontraremos um povo mais
amável''.

“Talvez fosse melhor irmos para a Espanha, a estação
já está adiantada; vamos fugir antes das chuvas e molhar só nossas
gargantas,'' disse um dos outros dois.

\quebra

“Guardei tudo na memória, como estão vendo. Depois disso, 
beberam uma caneca de aguardente cada um e adormeceram com a fronte
voltada para as estrelas. Eu, de início, tive vontade de
pedir que me levassem com eles e me ensinassem a tocar seus
instrumentos; mas não tive coragem, sem dúvida por ser sempre muito
difícil decidir"-se por qualquer coisa, e também porque tinha medo
de ser alcançado antes de estar fora da França.''

O ar pouco interessado dos outros três companheiros me fez pensar
que aquele pequeno já era um \textit{incompreendido}. Olhei atentamente para ele;
havia em seus olhos e sua fronte esse não sei quê precocemente fatal
que em geral afasta as simpatias e não sei por quê excitava a
minha, a ponto que tive por um instante a esquisita ideia de que 
poderia ter um irmão por mim mesmo desconhecido.

O sol já tinha se posto. A noite solene se instalara. As crianças se
separaram, indo cada qual, sem o saber, segundo as circunstâncias e
os acasos, amadurecer seu destino, escandalizar seus próximos e
gravitar em direção à glória ou à desonra.

\quebra\section[O tirso]{o tirso\protect\footnote{\uppercase{O} tirso, bastão envolto por pâmpanos e hera, com a extremidade em
forma de pinha, era atributo de \uppercase{B}aco, sendo usado pelas \uppercase{B}acantes.}}

\begin{flushright}
\textit{Para Franz Liszt}\protect\footnote{ Franz Liszt (1811"-1886), músico húngaro cuja obra encarna o ideal
poético de Baudelaire. Pouco se sabe sobre as relações entre os dois
artistas, que foram, no entanto, bastante estreitas.}
\end{flushright}\medskip

O que é um tirso? Segundo o senso moral e poético, é
um emblema sacro na mão dos sacerdotes e sacerdotisas ao celebrarem
a divindade da qual são intérpretes e servidores. Mas fisicamente é
apenas um bastão, um simples bastão, escora de lúpulo, tutor de
videira, seco, duro e reto. Em torno deste bastão, em meandros
caprichosos, brincam e se divertem hastes e flores, umas sinuosas e
esguias, outras curvadas como sinos ou taças viradas. E desta complexidade de linhas e cores, ternas ou
vibrantes, jorra uma auréola surpreendente. Não parece até que a linha curva e a espiral cortejam a
linha reta e dançam em torno dela em muda adoração? Não parece que
todas essas corolas delicadas, todos esses cálices, explosões de odores
e cores, executam um místico fandango em torno do bastão hierático? E
que imprudente mortal, contudo, ousará decidir se as flores e os
pâmpanos foram feitos para o bastão, ou se o bastão é só um pretexto
para mostrar a beleza dos pâmpanos e flores? O tirso, mestre poderoso e
venerado, caro Bacante\protect\footnote{  As Bacantes (em francês: bacchantes) eram sacerdotisas de Baco, deus do vinho, e celebravam as
bacanais. Baudelaire impõe aqui à palavra o gênero masculino (inexistente em francês e português) para designar Liszt.} da Beleza misteriosa e
apaixonada, é a
representação da sua espantosa dualidade. Jamais ninfa exasperada pelo invencível Baco sacudiu seu
tirso sobre a cabeça de suas companheiras alvoroçadas com a
energia e capricho com que você agita seu gênio sobre os corações de
seus irmãos. --- O bastão é sua vontade, reta, firme e inabalável; as
flores são o passeio

\quebra

\noindent{}da sua fantasia em torno à sua vontade; é o
elemento feminino executando em torno do macho suas prestigiosas
piruetas. Linha reta e linha arabesca, intenção e expressão, tensão da
vontade, sinuosidade do verbo, unidade do objetivo, variedade dos
meios, amálgama todo"-poderoso e indivisível do gênio, que análise
terá a detestável coragem de os dividir e separar?

Caro Liszt, através das névoas, para além dos rios, por cima das cidades
em que os pianos cantam sua glória, em que a imprensa traduz sua
sabedoria, onde quer que você esteja, nos esplendores da cidade eterna
ou nas névoas dos países sonhadores\protect\footnote{  A cidade eterna é Roma, 
onde Liszt costumava passar temporadas, e os
países sonhadores representam a Alemanha, onde sua música era muito
executada.} que Gambrinus\protect\footnote{  Gambrinus, rei 
legendário, contemporâneo de Carlos Magno, a quem é
atribuída a invenção da arte de fabricar cerveja.}
consola, improvisando cantos de deleite ou
de dor inefável, ou confiando ao papel suas meditações abstrusas,
chantre da Volúpia e Angústia eternas, filósofo, poeta e artista, eu o
saúdo em imortalidade!

\quebra\section[Embriaguem-se]{embriaguem-se}

Há que estar sempre embriagado. Tudo está nisto: é a única questão. Para
não sentir o terrível fardo do Tempo que lhes dilacera os ombros e os
encurva para a terra, embriagar"-se sem cessar é preciso.

Mas de quê? De vinho, poesia ou virtude, a escolha é sua. Mas
embriaguem"-se.

E se às vezes, na escadaria de um palácio, na verde relva de um
barranco, na solidão morna de seu quarto, vocês acordarem, com a
embriaguez já diminuída ou sumida, perguntem ao relógio, ao vento, à
vaga, à estrela, às aves, a tudo o que foge, a tudo o que geme, a tudo o que
rola, a tudo o que canta, a tudo o que fala, perguntem que horas são; e
o relógio, o vento, a vaga, a estrela, as aves responderão:
“É hora de embriagar"-se! Para não serem os escravos
martirizados do Tempo, embriaguem"-se! Sem cessar, embriaguem"-se! De
vinho, poesia ou virtude, a escolha é sua''.

\quebra\section[Já!?]{já!?}

Cem vezes já o sol tinha surgido, radiante ou entristecido, desta tina
imensa do mar cuja bordas mal se deixam avistar; cem vezes tinha
tornado a mergulhar, fulgurante ou melancólico, no seu banho imenso do
entardecer. Desde vários dias podíamos contemplar o outro lado do
firmamento e decifrar o alfabeto celeste dos antípodas. E cada um dos
passageiros gemia e resmungava. A proximidade da
terra parecia exasperar seu sofrimento. “Mas quando'',
 diziam, “vamos deixar de dormir
um sono chacoalhado pela vaga, perturbado por um vento que ronca mais
alto que nós? Quando vamos comer uma carne que não seja
salgada como o infame elemento que nos leva? Quando vamos poder
digerir numa poltrona parada?''

Alguns pensavam em seus lares, sentiam falta de suas mulheres
infiéis e aborrecidas e sua prole choramingas. Estavam todos tão
assustados com a imagem da terra ausente que teriam, acho, comido
grama com mais entusiasmo que os bichos.

Um litoral foi finalmente anunciado; e vimos, chegando perto, que era
uma terra magnífica, deslumbrante. Parecia desprender as músicas da vida
num vago murmúrio, e que daquelas costas, ricas em toda
espécie de verde, se exalava, a várias léguas, um delicioso cheiro de
flores e frutas.

Logo ficaram todos alegres, abdicaram do mau humor.
As rixas todas foram esquecidas, as mútuas injúrias, perdo-

\quebra

\noindent{}adas; os duelos combinados foram riscados da memória e os rancores voaram
feito fumaça.

Só eu estava triste, inconcebivelmente triste. Qual um padre a
quem se arrancasse a divindade, eu não podia, sem desolada amargura, me
desprender do mar tão monstruosamente sedutor, do mar tão
infinitamente diverso em sua assustadora simplicidade e que parece
conter em si, e representar com seus jogos, maneiras, raivas e
sorrisos, os humores, agonias e êxtases de todas as almas
que já viveram, vivem e haverão de viver!

Ao dar adeus àquela incomparável beleza, eu me sentia mortalmente abatido;
por essa razão, quando cada um dos meus companheiros disse:
“Enfim!'', eu não pude gritar senão:
“\textit{Já!?}'' 

Era, no entanto, a terra, a terra com seus ruídos, suas paixões, suas
comodidades, suas festas; era uma terra rica e magnífica, cheia de
promessas, que nos lançava um misterioso perfume de rosa e almíscar,
e de onde as músicas da vida nos alcançavam em amoroso murmúrio.

\quebra\section[As janelas]{as janelas}

Quem olha de fora por uma janela aberta não vê nunca
tanta coisa como quem olha para uma janela fechada. Não há objeto
mais profundo, misterioso, fecundo, mais tenebroso,
radiante, que uma janela aclarada por uma candeia. O que se pode ver
à luz do sol é sempre menos interessante que o que se passa por
detrás de uma vidraça. Nesse buraco negro ou luminoso vive a vida,
sonha a vida, sofre a vida.

Para além do ondular dos telhados, avisto uma mulher madura, já com
rugas, pobre, sempre debruçada sobre alguma coisa, e que nunca sai. Com
seu rosto, sua roupa, seu gesto, com quase nada refiz a
história desta mulher, ou melhor, sua lenda, e por vezes a conto a mim
mesmo chorando.

Tivesse sido um pobre homem velho, teria refeito a sua com igual
facilidade.

E me deito, feliz por ter vivido e sofrido em outros que não eu mesmo.

Talvez me digam: “Tem certeza de que esta lenda é
verdadeira?'' Que importa o que seja a realidade
situada fora de mim, se me ajudou a viver, a sentir que sou, e o que
sou?

\quebra\section[O desejo de pintar]{o desejo de pintar}

Infeliz pode ser o homem, mas feliz é o artista que o desejo dilacera!

Ando louco para pintar aquela que tão raro apareceu e tão
depressa fugiu, como algo belo e saudoso atrás do viajante transportado
noite adentro. Há tanto tempo já, que ela desapareceu!

Ela é bela, e mais que bela: é surpreendente. O negro nela excede, e
tudo que ela inspira é noturno e profundo. Seus olhos são dois antros
em que cintila vagamente o mistério, e seu olhar ilumina qual um raio: é
uma explosão dentro das trevas.

Eu a compararia a um sol negro, se pudéssemos conceber um astro negro
vertendo luz e felicidade. Mas ela lembra mais naturalmente a
lua, que sem dúvida a marcou com sua temível influência; não a lua
branca dos idílios, semelhante a uma noiva fria, mas a lua
sinistra e inebriante, suspensa no fundo de uma noite tempestuosa e
atropelada pelas nuvens apressadas; não a lua mansa e discreta a
visitar o sono dos puros, mas a lua arrancada do céu, vencida e
revoltada, que as Feiticeiras tessálias obrigam duramente a dançar
na relva apavorada!

Na sua fronte pequena residem a vontade tenaz e o amor da presa. Entretanto, na base deste rosto inquietante, em que narinas moventes aspiram
o desconhecido e o impossível, rebenta, com graça indizível, o riso de
uma boca grande, vermelha e branca, e deliciosa, que faz sonhar com o
milagre de uma flor deslumbrante desabrochando em terreno vulcânico.

Há mulheres que inspiram a vontade de vencê"-las e desfrutá"-las; mas
esta dá o desejo de morrer lentamente sob seu olhar.

\quebra\section[Os favores da lua]{os favores da lua}

A lua, que é o capricho em si, olhou pela janela enquanto você dormia
em seu berço, e pensou: “Gosot desta criança''.

E ela desceu suavemente sua escadaria de nuvens e passou sem ruído
através das vidraças. Então se estendeu sobre você com a branda ternura 
de uma mãe e pôs cores dela no seu rosto. Assim suas pupilas se
tornaram verdes e suas faces extraordinariamente pálidas. Foi
contemplando essa visitante que seus olhos tão estranhamente se
dilataram; e tão carinhosamente ela apertou-lhe a garganta que você
conservou para sempre a vontade de chorar.

Entretanto, na sua expansão de alegria, a Lua preenchia o quarto inteiro 
qual uma atmosfera fosfórica, um veneno luminoso; e aquela 
luz viva toda pensava e dizia: “Você há de sofrer eternamente
a influência do meu beijo. Há de ser bela à minha maneira. Há de amar
aquilo que eu amo e aquilo que me ama: a água, as nuvens, o silêncio e
a noite, o mar imenso e verde, a água informe e multiforme; o lugar em
que não estiver; o amante que não conhecer; as flores monstruosas, os
aromas que causam delírio; os gatos que se pasmam sobre os pianos e
gemem, como as mulheres, com uma voz rouca e doce!

E você será amada por meus amantes, cortejada por meus
cortesãos. Será a rainha dos homens de olhos verdes cuja
garganta também apertei em minhas carícias noturnas; daqueles que amam
o mar, o mar imenso, tumultuoso e verde, a água informe e multiforme, o
lugar em que não \linebreak

\quebra

\noindent{}estão, a mulher que não conhecem, as flores sinistras
que lembram incensórios de uma religião ignorada, os
aromas que perturbam a vontade, e os animais selvagens e voluptuosos
que são os emblemas da sua loucura.''

E é por isto, maldita querida criança mimada, que aqui estou deitado
aos seus pés, buscando em toda a sua pessoa o reflexo da temível
Divindade, da fatídica madrinha, da ama que envenena todos os
\textit{lunáticos}.

\quebra\section[Qual será a verdadeira?]{qual será a verdadeira?}

Conheci uma certa Benedicta, que enchia a atmosfera de ideal, e cujos
olhos espalhavam o desejo da grandeza, beleza, glória e tudo o
que faz crer na imortalidade.

Mas esta moça milagrosa era bonita demais para viver muito tempo;
assim, morreu poucos dias depois que a conheci, e fui eu
mesmo quem a sepultou, num dia em que a primavera agitava seu
incensório até dentro dos cemitérios. Fui eu quem a sepultou, bem
fechada num esquife de madeira perfumada e incorruptível como os baús
indianos.

E enquanto meus olhos se mantinham grudados no lugar onde estava enterrado
o meu tesouro, avistei de súbito uma pessoinha singularmente parecida com a defunta que, espezinhando a terra fresca com 
estranha e histérica violência, dizia, dando risada:
A verdadeira Benedicta sou eu! Sou eu, uma bela
canalha! E como castigo por sua loucura e cegueira, você há de me amar assim
como sou!

Mas eu, furioso, respondi: “Não! Não! Não!''
E para melhor acentuar minha recusa, espezinhei a
terra com tanta violência que minha perna afundou até o joelho na sepultura recente e, feito
um lobo preso na armadilha, estou atado, quem sabe para sempre, à fossa
do ideal.

\quebra\section[Um cavalo de raça]{um cavalo de raça}

Ela é um bocado feia. No entanto, é deliciosa!

O Tempo e o Amor a marcaram com suas garras e lhe ensinaram cruelmente o
que cada minuto e cada beijo levam consigo de juventude e frescor.

Ela é realmente feia; é formiga, aranha, o que quiserem, esqueleto até;
mas é também beberagem, magistério, bruxaria! Em suma, é uma delícia.

O Tempo não conseguiu romper a harmonia crepitante do seu andar nem a
inabalável elegância de sua estrutura. O Amor não alterou a suavidade
do seu hálito de criança, e o Tempo nada arrancou da sua abundante
crina de onde se exala em selvagens perfumes toda a
endiabrada vitalidade do sul da França: Nîmes, Aix, Arles, Avignon, Narbonne,
Toulouse, abençoadas cidades do sol, charmosas e enamoradas!

O Tempo e o Amor em vão a morderam a plenos dentes; em nada
diminuíram o charme vago, mas eterno, do seu peito viril.

Desgastada, talvez, mas não cansada, e sempre heroica, lembra os cavalos
de grande raça, que o olhar do autêntico amador reconhece mesmo quando
atrelados a uma carruagem de aluguel ou pesada carroça.

Além disso, é tão doce e fervorosa! Ama como se ama no outono;
a proximidade do inverno parece acender no seu coração um
fogo novo, e o servilismo do seu carinho nunca é em nada cansativo.

\quebra\section[O espelho]{o espelho}

Entra um homem horrendo e se olha no espelho.

“Por que olhar"-se no espelho se nele só vai se
ver com desprazer?''

O homem horrendo me responde: “Meu senhor, segundo os imortais
princípios de 89, todos os homens são iguais em direitos; possuo,
portanto, o direito de mirar"-me; se com prazer ou desprazer, só diz
respeito à minha consciência''.

Pelo bom"-senso, sem dúvida, eu estava certo; do ponto de vista
da lei, porém, ele não estava errado.

\quebra\section[O porto]{o porto}

Um porto é uma morada encantadora para uma alma cansada das lides da
vida. A amplitude do céu, a arquitetura móvel das nuvens, as colorações
mutantes do mar, o cintilar dos faróis, são um prisma maravilhosamente
próprio para divertir os olhos sem jamais enfastiá"-los. As formas
esbeltas dos navios, de enxárcia complicada, nos quais o marulho
imprime oscilações harmoniosas, servem para cultivar na alma o gosto
do ritmo e da beleza. E existe também, sobretudo, uma espécie de prazer
misterioso e aristocrático, para aquele que já não tem curiosidade
ou ambição, em contemplar, deitado no belvedere ou apoiado no
quebra-mar, os movimentos todos daqueles que partem e 
retornam, daqueles que ainda possuem a força de querer, o desejo de
viajar ou de enriquecer.

\quebra\section[Retratos de amantes]{retratos de amantes}

Numa saleta para homens, ou seja, numa sala de fumar adjacente a um
elegante bordel, quatro homens fumavam e bebiam. Não eram
exatamente jovens, nem velhos, nem belos, nem feios; mas, velhos ou
jovens, traziam essa distinção, não imperceptível, dos veteranos da
alegria, esse indescritível não sei quê, essa tristeza fria e zombeteira
que diz claramente: “Já vivemos intensamente e
estamos em busca de algo para amar e estimar''.

Um deles puxou a conversa para o assunto das mulheres. Teria sido mais
filosófico sequer tocar nele; mas existem pessoas de espírito que,
depois de beber, não desprezam as conversas banais. Escuta"-se, então,
quem está falando como se escutaria música de dança.

“Todos os homens'', dizia, “ já tiveram a idade de
Querubim:\protect\footnote{ O Querubim é o anjo que, na primeira hierarquia, ocupa o segundo
lugar, sendo representado nas Artes com corpo de criança.}
 é a época em que, na falta de
dríades,\protect\footnote{ As dríades eram, na mitologia grega, 
as ninfas dos bosques.} abraçamos, sem desgostar, o tronco dos
carvalhos. É o primeiro nível do amor. No segundo nível, começamos a
escolher. Poder deliberar já é em si uma decadência. É então que procuramos
decididamente a beleza. Quanto a mim, senhores, orgulho-me de ter
chegado, há muito tempo, na época climatérica do terceiro nível, quando
a beleza em si já não basta quando não temperada pelo perfume, pelo
adorno, \textit{et cetera}. Confesso até que aspiro, às vezes, como que a uma
felicidade desconhecida, a um quarto nível que deve indicar a calma
absoluta. Durante toda a minha vida, porém, exceto na idade de Querubim,
fui mais sensível que qualquer outro à enervante tolice, à irritante
mediocridade das mulheres. O que mais gosto nos animais é a sua
candura. Pois avaliem o quanto não sofri com a minha última
amante.

Era a bastarda de um príncipe. Bela, é escusado dizer; se
não, por que a teria tomado? Mas estragava essa grande qualidade
com uma ambição inconveniente e disforme. Era uma mulher que sempre
queria bancar o homem. “Você não é homem! Ah! Se eu
fosse homem! De nós dois, o homem sou eu!'' Tais eram os
insuportáveis refrões que saíam daquela boca, de onde eu queria que surgissem apenas canções. A respeito de um livro, um poema, uma
ópera pela qual eu deixava transparecer minha admiração:
“Então você acredita que isto tem força?'' ela logo dizia; “será que você
entende de força?'', e argumentava.

Um belo dia passou a se dedicar à química; de modo que
entre a minha boca e a dela encontrei a partir de então uma máscara de vidro.
Além do mais, melindrosíssima. Se eu, por vezes, a atropelava com um
gesto meio apaixonado demais, ela se convulsava feito uma sensitiva
violada\ldots\ 

--- E como é que isso terminou? disse um dos outros três. Não sabia que
você era tão paciente.

--- Deus – ele prosseguiu --- trouxe ao mal o remédio. Um dia encontrei esta
Minerva, faminta de força ideal, num particular com meu empregado, e
numa situação que me obrigou a me retirar discretamente para não
fazê"-los corar. À noite despedi os dois, pagando os atrasados dos seus
honorários.

\quebra

--- Quanto a mim --- retomou o interruptor ---, só posso queixar"-me de mim
mesmo. A felicidade veio morar em minha casa, e eu não a reconheci. O
destino me concedera, nos últimos tempos, desfrutar de uma
mulher que era seguramente a mais doce, a mais submissa e a mais
dedicada das criaturas, e sempre pronta! E sem entusiasmo!
“Está bem, já que é do seu agrado.'' Era
sua resposta costumeira. Se você enchesse esta parede ou
este sofá de pauladas, arrancaria mais suspiros do que arrancavam, do seio de
minha amante, os ímpetos do amor mais exaltado. Depois de um ano de
vida em comum, ela me confessou que nunca conhecera o prazer.
Desgostei"-me com o duelo desigual, e aquela moça incomparável se
casou. Tive mais tarde o desejo de revê"-la, e ela me disse,
mostrando"-me seis lindas crianças: “Pois é, caro
amigo, a esposa ainda é tão virgem quanto era sua
amante.'' Nada estava mudado naquela pessoa. Às vezes,
sinto sua falta: deveria ter me casado com ela.

Os outros puseram"-se a rir, e um terceiro disse por sua vez:

Senhores, conheci gozos que talvez tenham negligenciado.
Falo da comicidade no amor, e uma comicidade que não exclui a
admiração. Admirei mais minha última amante do que vocês conseguiram,
acho eu, odiar ou amar as suas. E todo o mundo a admirava como
eu. Quando entrávamos num restaurante, passados alguns minutos todos
esqueciam de comer para contemplá"-la. Os próprios garçons e a senhora
do balcão sentiam esse êxtase contagioso a ponto de esquecerem suas
obrigações. Em suma, vivi por algum tempo com um \textit{fenômeno}
vivo. Ela comia, mastigava, triturava, devorava, engolia, mas com o
jeito mais leve e despreocupado do mundo. Ela me manteve assim 
muito tempo em êxtase. Tinha uma maneira suave, sonhadora, inglesa
e romanesca de dizer: “Estou com fome!'' E
repetia essas palavras dia e noite, mostrando os dentes mais bonitos do
mundo, que os teriam, a um só tempo, comovido e divertido. Eu
poderia ter feito fortuna exibindo"-a pelas feiras como \textit{monstro
polifágico}. Eu a alimentava bem; e no entanto, ela me deixou\ldots\ 

--- Por um fornecedor de alimentos, sem dúvida?

--- Por algo aproximado, uma espécie de empregado da intendência que,
por algum passe de mágica dele conhecido, talvez forneça à pobre
menina a ração de vários soldados. Foi, pelo menos, o que imaginei.

--- Quanto a mim, --- disse o quarto --- aguentei sofrimentos atrozes pelo motivo oposto do que em geral se censura à egoísta fêmea. Acho sem fundamento vocês, mortais de muita sorte, se queixarem de imperfeições das suas
amantes!”

Aquilo foi dito num tom muito sério, por um homem de aspecto suave e
tranquilo, fisionomia quase clerical, infelizmente iluminada por
olhos de um cinza claro, desses olhos cujo olhar diz: “Eu
quero!'' ou “É preciso.'' ou
então: “Eu nunca perdoo!''

Se, nervoso como o conheço, G\ldots, covardes e levianos
como são os dois, K\ldots\  e J\ldots\ , vocês tivessem sido acoplados a 
certa mulher que conheço, teriam fugido ou morrido. Eu, como
estão vendo, sobrevivi. Imaginem uma pessoa incapaz de cometer um erro
de sentimento ou de cálculo; imaginem uma aflitiva serenidade de
caráter, uma dedicação sem fingimento e sem ênfase, uma doçura sem
fraqueza, uma energia sem violência. A história do meu amor lembra
uma interminável viagem por uma superfície pura e polida como um
espelho, vertiginosamente monótona, que refletisse todos os meus
sentimentos e gestos com a irônica exatidão da minha própria
consciência, de modo que eu não podia me permitir um gesto ou um
sentimento desarrazoado sem perceber imediatamente a muda censura de
meu inseparável espectro. O amor me parecia uma tutela. Quantas
tolices ela impediu que eu fizesse, que lamento não ter cometido!
Quantas dívidas pagas contra a minha vontade! Ela me privava de todos
os benefícios que eu poderia tirar da minha loucura particular!
Com uma fria e intransponível regra, barrava todos os meus
caprichos. Para cúmulo de horror, passado o
perigo não exigia gratidão. Quantas vezes me segurei para não saltar"-lhe ao pescoço,
gritando: “Seja imperfeita, miserável! Para que
eu possa amá"-la sem mal"-estar e sem raiva.'' Durante
vários anos admirei"-a, com o coração cheio de ódio. Por fim, não fui eu
quem morreu!

--- Ah! --- fizeram os outros --- Então ela morreu?

--- É. Aquilo não podia continuar assim. O amor tinha se tornado para mim
um pesadelo opressivo. Vencer ou morrer, como se diz na Política, era a
alternativa que o destino me impunha! Certa noite, num bosque\ldots\  à
beira de um \linebreak

\quebra

\noindent{}charco\ldots\  depois de um passeio melancólico em que os
olhos, os dela, refletiam a suavidade do céu, e o coração, o
meu, estava crispado feito o inferno\ldots\ 

--- O quê!

--- Como!

--- O que quer dizer?

--- Era inevitável. Tenho demasiado sentimento de equidade para
surrar, ultrajar ou despedir um empregado incensurável. Mas precisava conciliar o sentimento com o horror que aquele ser me 
inspirava; livrar"-me daquele ser sem lhe faltar ao respeito. 
O que queriam que eu fizesse com ela, \textit{se ela era
perfeita}?

Os três outros companheiros olharam para ele com um olhar vago e
ligeiramente estonteado, como fingindo não compreender, e como 
confessando implicitamente que não se sentiam, quanto a eles, capazes
de uma ação tão rigorosa, se bem que, por sinal, suficientemente
explicada.

Em seguida, mandaram trazer mais garrafas, para matar o Tempo que é tão difícil de morrer, e apressar a Vida que corre tão devagar.

\quebra\section[O galante atirador]{o galante atirador}

O carro atravessava o bosque, quando ele o fez parar à proximidade de um
tiro, dizendo que seria agradável atirar umas balas para \textit{matar} o
Tempo. Matar este monstro não é a ocupação mais trivial e legítima de
toda pessoa? --- E ofereceu galantemente a mão à sua cara, deliciosa e
execrável mulher, a misteriosa mulher a quem deve tantos prazeres,
tantas dores, e talvez também grande parte do seu gênio.

Várias balas bateram longe do alvo proposto; uma delas chegou a cravar"-se
no teto; e como a encantadora criatura ria loucamente, zombando da
inabilidade do esposo, este voltou"-se bruscamente para ela e lhe
disse: “Repare naquela boneca, ali, à direita, com
nariz arrebitado e um jeito tão altivo. Pois bem, caro anjo, \textit{faço de
conta que é você}''. E ele fechou os olhos, e puxou o
gatilho. A boneca foi nitidamente decapitada.

Então, inclinando"-se para a sua cara, deliciosa, execrável mulher, sua
incontornável e impiedosa Musa, e beijando"-lhe respeitosamente a mão,
acrescentou: Ah! meu caro anjo, que grato lhe sou por
minha habilidade!

\quebra\section[A sopa e as nuvens]{a sopa e as nuvens}

Minha louquinha bem"-amada me dava de almoçar, e pela janela aberta da
sala de jantar eu contemplava as moventes arquiteturas que Deus cria com
os vapores, as maravilhosas construções do impalpável. E pensava,
em meio à minha contemplação: “Todas estas
fantasmagorias são quase tão belas como os olhos de minha
bem"-amada, a monstruosa maluquinha de olhos verdes.”

E de repente levei um violento soco nas costas, e ouvi uma voz rouca e
encantadora, uma voz histérica e como enrouquecida de aguardente,
a voz da minha querida bem"-amada, que dizia: “Você
vai ou não vai tomar logo a sua sopa, seu safado de um mercador de
nuvens?''\protect\footnote{ “Seu safado de um mercador de nuvens'' 
procura traduzir a expressão “Sacré bougre de marchand de
nuages'', constante do manuscrito original e reduzida a
iniciais na edição de 1869.}

\quebra\section[O tiro e o cemitério]{o tiro e o cemitério}

--- \textit{À vista do cemitério, Botequim.} --- “Singular tabuleta, ---
pensou nosso andarilho, --- mas, de fato, feita para dar sede! Não há dúvida, o dono desta taberna sabe apreciar Horácio e os poetas alunos
de Epicuro.\protect\footnote{  Filósofo grego (séc. \textsc{iii} a.C.) cuja doutrina, que
visava o prazer oriundo do espírito e da prática da virtude, acabou
deixando de seu autor a imagem de um libertino. Horácio, poeta romano
(séc. \textsc{i} a.C.), proclamava"-se um dos seus discípulos.} Talvez até conheça o
profundo refinamento dos antigos egípcios, para quem não havia um bom festim sem um
esqueleto, ou sem um emblema qualquer da brevidade da
vida.

E entrou, tomou um copo de cerveja frente aos túmulos e fumou
lentamente um charuto. Então lhe veio a fantasia de descer ao
cemitério, de relva tão alta e tão convidativa, e onde reinava um
sol tão rico.

Com efeito, luz e calor ali assolavam furiosamente, e até parecia que o sol embriagado se espreguiçava inteiro num tapete de
flores magníficas, adubadas pela destruição. Um imenso sussurro de vida
enchia o ar --- a vida dos infinitamente pequenos ---, interrompido a
intervalos regulares pelo crepitar dos disparos de um tiro próximo, 
feito o estouro de rolhas de champanha no burburinho de uma
sinfonia em surdina.

Então, sob o sol que lhe esquentava o cérebro e na atmosfera dos
ardentes aromas da Morte, ele ouviu uma voz cochichar sob o túmulo em
que estava sentado. E esta voz dizia: ``Malditas sejam
suas miras e suas carabinas, vivos turbulentos, que tão
pouco se importam com os defuntos e com seu repouso divino! Malditas sejam as
ambições, malditos sejam seus cálculos, mortais impacientes que vêm
estudar a arte de matar junto ao santuário da Morte! Se soubessem
como o prêmio é fácil de ganhar, como o alvo é fácil de acertar, e o
quanto tudo é nada, exceto a Morte, vocês não se cansariam tanto, vivos
laboriosos, e perturbariam com menos frequência o sono dos que há
muito atingiram o Alvo, único alvo verdadeiro desta vida detestável!

\quebra\section[A perda de auréola]{perda de auréola}

--- O quê!? Você por aqui, meu caro? Num lugar suspeito?
Você, o bebedor de quintessências? O comedor de ambrosia? Na
verdade, tenho de surpreender"-me!

--- Você conhece, caro amigo, meu pavor pelos cavalos e pelos carros. Ainda
há pouco, quando atravessava a avenida, apressadíssimo, e
saltitava na lama em meio a esse caos movediço em que a morte chega a
galope por todos os lados ao mesmo tempo, minha auréola, num movimento
brusco, escorregou da minha cabeça para a lama da calçada. Não tive
coragem de juntá"-la. Julguei menos desagradável perder minhas
insígnias do que deixar que me rompessem os ossos. E depois, pensei, há
males que vêm para bem. Posso agora passear incógnito, praticar ações
vis e me entregar à devassidão, como os simples mortais. E 
aqui estou, igualzinho a você, como vê!

--- Você deveria ao menos mandar pôr um anúncio pela auréola, ou mandar
reavê"-la pelo delegado.

--- Não, ora essa! Sinto-me bem aqui. Só você me reconheceu. A dignidade, aliás, me entedia. E também, me alegra pensar que algum poeta ruim
há de juntá"-la e vesti"-la impudentemente. Fazer alguém feliz, que
prazer! Principalmente um feliz que ainda vai me fazer rir! Pense em X ou em
Z, puxa! Que engraçado vai ser!

\quebra\section[Senhorita bisturi]{senhorita bisturi}

Eu estava chegando ao final do subúrbio, sob os clarões do gás,
quando senti um braço se insinuando suavemente sob o meu e ouvi uma
voz que cochichava: “Moço, o senhor é
médico?''

Olhei: era uma jovem alta, forte, com olhos muito abertos, ligeiramente
maquiada, os cabelos esvoaçando junto às fitas do seu gorro.

“Não, não sou médico.'' Deixe"-me passar.'' “Oh! Claro que
o senhor é médico, estou vendo. Vamos até a minha casa. Vamos, o senhor vai ficar
satisfeito comigo!'' “Sem dúvida, irei visitá"-la, mas mais
tarde, \textit{depois do médico}, que diabos!\ldots\ '' “Ah! Ah!'', ela fez, ainda
pendurada em meu braço, e caindo na gargalhada, “o senhor é um médico
brincalhão, conheci vários deste tipo. Venha!''

Amo apaixonadamente o mistério, pois sempre tenho a esperança de
elucidá"-lo. Deixei"-me, portanto, arrastar por aquela companheira,
ou melhor, aquele enigma inesperado.

Omito a descrição do casebre; pode ser encontrada em vários antigos
poetas franceses bem conhecidos. Só que, detalhe não percebido por
Régnier,\protect\footnote{  Mathurin Régnier (1573"-1613), poeta conhecido por sua sátira ágil,
muito lido por Baudelaire.} dois ou três retratos de médicos famosos
estavam pendurados nas paredes.

Como fui paparicado! Um bom fogo, vinho quente, charutos; e ao me
oferecer estas coisas boas e ela própria acendendo um charuto, a bufona
criatura me dizia: “Esteja em casa, caro amigo, fique à
vontade. Isto vai lhe lembrar o hospital e os bons tempos da juventude. 
Ora essa! Onde arranjou esses
cabelos brancos? O senhor não era assim, ainda pouco tempo atrás,
quando era residente de L\ldots\  Lembro que era o senhor quem o assistia nas
cirurgias graves. Está aí um homem que gosta de cortar, entalhar e
roer! Era o senhor quem lhe alcançava os instrumentos, os fios e as
esponjas. E com que orgulho ele dizia, terminada a cirurgia,
olhando o relógio: “Cinco minutos,
senhores!'' Ah! Eu ando por toda parte. Conheço bem
esses senhores.''

Instantes mais tarde, tratando"-me por você, retomava a sua
cantilena, e me dizia: “Você é médico, não é, meu
gato?''.

Esse ininteligível refrão fez com que eu me erguesse num salto.
``Não!'' gritei, furioso.

--- Cirurgião, então?

--- Não! Não! A não ser que seja para cortar a sua cabeça! Santo
sagrado cibório de Santa Caftina!\protect\footnote{   ``Santo sagrado Cibório de Santa Caftina''
procura traduzir a expressão ``Sacré saint ciboire de
sainte maquerelle'', constante do manuscrito original e
reduzida a iniciais na edição de 1869.}

--- Espere, replicou, você vai ver.

E tirou de um armário um maço de papéis, que nada mais era que a
coleção de retratos dos médicos ilustres daquele tempo, litografados
por Maurin, que esteve exposta durante muitos anos no Quai
Voltaire.

--- Olhe! Reconhece esse aqui?

--- Sim, é X. O nome está embaixo, aliás; mas eu o conheço
pessoalmente.

--- Eu sabia! Olhe! Esse é Z., aquele que dizia na sua aula,
referindo"-se a X.: “Esse monstro que traz no rosto a
negritude da própria alma!''. Só porque o outro não
partilhava a opinião dele sobre um caso! Como ríamos disso na

\quebra

\noindent{}Escola, na época! Você se lembra? Olhe! Esse é K., que
denunciava ao governo os insurretos que tratava em seu hospital. Era o
tempo das rebeliões. Como será possível um homem tão bonito ter tão
pouco coração? Agora, esse é W., um famoso médico inglês; eu o
peguei quando viajou para Paris. Parece uma donzela, não é
mesmo?

E, como eu tocasse num pacote atado com barbante que também estava em cima da
mesinha: “Espere um pouco'', ela disse, “aqui estão os residentes internos, e nesse pacote estão os externos''.

E ela espalhou em leque um monte de imagens fotográficas 
representando fisionomias bem mais jovens.

--- Quando nos virmos de novo, você vai me dar um retrato seu,
não é, querido?

--- Mas --- disse eu, também seguindo, por minha vez, minha ideia fixa ---,
por que é que você acha que eu sou médico?

--- É que você é tão afável e tão bom com as mulheres!

--- Que lógica singular!, pensei comigo mesmo.

--- Oh! Eu nunca me engano; conheci um bom número deles. Gosto tanto
desses senhores que, apesar de não estar doente, às vezes vou vê"-los,
só por vê"-los. Alguns me dizem friamente: “Você não
está nem um pouco doente!'' Mas há outros que me
compreendem, porque consigo seduzi-los.

--- E quando não a compreendem?

--- Ora! Já que os incomodei \textit{inutilmente}, deixo dez francos sobre a
lareira. São tão bons e tão doces, esses homens!

\quebra

\noindent{}Descobri no
Piété\protect\footnote{  \textit{Hôpital de la Piéte}, em Paris.}
 um residentezinho, lindo como um anjo
e tão bem-educado! E como trabalha, pobre rapaz! Seus colegas me
disseram que ele não tem dinheiro porque os pais são uns pobres que
não podem lhe mandar nada. Isto me deu confiança. Afinal, sou uma
mulher bastante bonita, apesar de não muito jovem. Eu disse a ele:
“Venha me visitar, venha me visitar seguidamente. E
comigo, não faça cerimônia; não preciso de dinheiro''. Mas
você compreende que eu dei a entender isso por uma série de
maneiras; não disse assim cruamente; tinha tanto medo de
humilhar o pobre merino! Pois bem! Você acredita que eu tenho um
desejo esquisito que não ouso dizer a ele? Queria que ele viesse
me visitar com sua maleta e seu jaleco, até com um pouco de sangue em
cima!

Ela disse isto com tanta candura, como um homem sensível diria a
uma atriz que ele amasse: “Quero vê"-la vestida com o
traje que você usava naquele famoso papel que criou''.

Eu, persistindo, repliquei: “Você consegue lembrar da época
em que surgiu em você essa paixão tão singular?''.

Foi difícil fazer-me entender; consegui, finalmente. Mas ela então respondeu
com um ar muito triste, e até, tanto quanto me lembre, desviando o
olhar: “Eu não sei.. não lembro''.

Que esquisitices não encontramos numa cidade grande, quando sabemos
passear e olhar? A vida fervilha de monstros inocentes. Senhor, meu
Deus! Vós, o Criador, vós, o \linebreak

\quebra

\noindent{}Mestre; vós que fizestes a Lei e a
Liberdade; vós, Soberano que deixais acontecer, vós, juiz que perdoais;
vós que estais repleto de motivos e causas e que talvez tenhais
colocado o gosto do horror em meu espírito a fim de converter meu
coração, feito a cura na ponta de uma lâmina! Senhor, tende piedade,
tende piedade dos loucos e das loucas! Oh, Criador! Acaso podem existir
monstros aos olhos d’Aquele único que sabe por
que eles existem, como eles \textit{se fizeram} e como poderiam não se
terem feito?

\quebra\section[Any where out of the world, qualquer lugar fora do mundo]{any where out of the world,
\protect\footnote{\uppercase{O} título \textit{\uppercase{A}ny where 
out of the world} é uma citação do poema
``\uppercase{B}ridge of \uppercase{S}ighs'', de \uppercase{T}homas \uppercase{H}ood, também citado por \uppercase{E}dgar \uppercase{A}. \uppercase{P}oe em seu ``\uppercase{P}oetic \uppercase{P}rinciple''.} 
qualquer lugar fora do mundo}

Esta vida é um hospital em que cada doente está possuído pelo desejo de
mudar de leito. Esse queria sofrer diante da estufa e aquele
acredita que iria se curar do lado da janela.

Parece-me que eu sempre estaria bem onde não estou, e essa questão
da mudança é uma das que discuto sem cessar com minha alma.

“Diga-me, alma minha, pobre alma arrefecida, o que você
diria de morar em Lisboa? Deve fazer calor por lá, e você poderia se revigorar
feito um lagarto. A cidade fica à beira d’água; dizem que é
construída em mármore, e que o povo tem tal ódio dos vegetais que
arranca todas as árvores. Está aí uma paisagem ao seu gosto;
paisagem feita de luz e mineral, e líquido para refleti"-los.''

Minha alma não responde.

“Já que aprecia tanto o descanso com o espetáculo do
movimento, quer ir morar na Holanda, essa terra beatífica? Você
talvez se distraia nessa região cuja imagem seguidamente admirou nos
museus. O que diria de Rotterdam, você que tanto gosta das florestas de
mastros, e dos navios atracados junto às casas?''

Minha alma permanece muda.

“Batávia\footnote{ Batávia (hoje Jacarta), capital da Indonésia.}
 talvez lhe conviesse melhor?
Lá, aliás, encontraríamos o espírito da Europa unido à beleza
tropical.''

Nenhuma palavra. Minha alma estaria morta!?

“Você teria então chegado a tal ponto de torpor que só se
apraz em sua própria dor? Se é assim, fujamos para os países que são
analogias da Morte! Tenho a nossa solução, pobre alma! Faremos as
malas para Tornio.\protect\footnote{  Tornio, cidade da Finlândia.} Vamos mais longe ainda, ao ponto
extremo do Báltico; mais longe ainda da vida, se possível:
instalemo"-nos no polo. Lá o sol não faz mais que roçar obliquamente a terra, e as
lentas alternâncias entre a luz e a noite suprimem a variedade e aumentam a
monotonia, essa metade do nada. Lá, vamos poder tomar longos banhos de
trevas enquanto, para nos distrair, as auroras boreais nos
enviarão de quando em quando seus feixes rosados, como reflexos de um fogo
de artifício do Inferno!''

Enfim, minha alma explode e, sabiamente, me grita:

“Qualquer lugar! Qualquer lugar! Desde que fora deste
mundo!''

\quebra\section[Espanquemos os pobres!]{espanquemos os pobres!}

Eu tinha, durante quinze dias, me confinado em meu quarto e me cercado dos
livros da moda naquela época (dezesseis, dezessete anos atrás); quero dizer, dos
livros em que é tratada a arte de tornar os povos felizes, sábios e
ricos em vinte e quatro horas. Tinha, pois, digerido --- engolido,
quero dizer --- todas as elucubrações de todos esses empreiteiros da
felicidade pública --- os que aconselham todos os pobres a se
fazerem escravos, e os que os convencem de que são todos reis
destronados. Ninguém há de estranhar que eu me encontrasse então num
estado de espírito que beirava a vertigem ou estupidez.

Tinha a impressão de sentir apenas, confinado no fundo do meu intelecto, o
gérmen obscuro de uma ideia superior a todas as receitas de comadre 
cujo dicionário eu havia recentemente percorrido. Não passava, porém, da 
ideia de uma ideia, algo infinitamente vago.

E saí, com muita sede. Pois o gosto apaixonado das más leituras gera
uma necessidade proporcional de ar puro e refresco.

Ia entrando numa taberna, quando um mendigo me estendeu o chapéu, com
um desses olhares inesquecíveis que derrubariam os tronos, se o
espírito revolvesse a matéria e se o olhar de um magnetizador pudesse
amadurecer as uvas.

Ao mesmo tempo, ouvi uma voz que sussurrava em meu ouvido; uma voz que
reconheci muito bem: a voz de um Anjo bom, ou um bom Demônio, que me
acompanha em toda parte. Se Sócrates tinha o seu bom Demônio, por que
eu não teria o meu Anjo bom, e por que eu não teria, como
Sócrates, a honra de obter meu certificado de loucura, assinado pelo
sutil Lelut ou o tão sensato Baillarger?\protect\footnote{  Lelut e Baillarger, 
médicos contemporâneos de Baudelaire que explicavam a 
genialidade através da loucura. O primeiro escrevera um
artigo demonstrando a loucura de Sócrates.}

Existe, entre o Demônio de Sócrates e o meu, a diferença que o de Sócrates
só se manifestava a ele para proibir, avisar, impedir, e que o meu se
digna aconselhar, sugerir, persuadir. O pobre Sócrates só tinha um
Demônio proibidor; o meu é um grande afirmador, o meu é um Demônio de
ação, ou um Demônio de combate.

Ora, eis o que sua voz me cochichava: “Só é o igual do
outro quem pode prová-lo, e só é digno da liberdade quem sabe
conquistá"-la''.

Imediatamente, me joguei sobre o meu mendigo. Com um murro só lhe acertei
um olho que, num instante, ficou do tamanho de uma bola. Quebrei
uma unha ao rebentar dois dos seus dentes, e como não me sentisse forte o
bastante, tendo nascido delicado e pouco tendo me exercitado no boxe,
para espancar rapidamente o velhinho, agarrei"-o com uma mão pelo
colarinho, com a outra segurei seu pescoço, e me pus a bater
vigorosamente sua cabeça contra um muro. Devo confessar que eu tinha
previamente, com um olhar, examinado os arredores e concluído que
naquele subúrbio deserto, estaria por um bom tempo fora do
alcance de qualquer policial.

Derrubando em seguida o fraco sexagenário com um pontapé desfechado em suas costas, enérgico o
bastante para romper-lhe as omoplatas, agarrei um galho grande de árvore que estava jogado no
chão e o surrei com a energia obstinada dos cozinheiros querendo
amaciar um bife.

De repente --- oh, milagre! Oh, gozo do filósofo que comprova a
excelência de sua teoria! --- vi a vetusta carcaça se virar,
erguer"-se com uma energia que eu jamais teria suspeitado numa máquina
tão singularmente avariada e, com um olhar de ódio que me pareceu de
\textit{bom augúrio}, o patife decrépito se atirou em cima de mim, me contundiu
os dois olhos, quebrou"-me quatro dentes, e com o mesmo galho de
árvore me moeu de pancadas. Com a minha enérgica medicação, portanto, eu 
tinha lhe devolvido o orgulho e a vida.

Fiz então uns tantos sinais dando a entender que considerava
finda a discussão e, reerguendo-me com a satisfação de um sofista do
Pórtico, disse"-lhe: “O senhor, o senhor é meu igual!
Queira me dar a honra de dividir minha bolsa comigo; e lembre"-se, se for
realmente um filantropo, que é preciso aplicar em todos
os seus companheiros, quando lhe pedirem esmola, a teoria que tive a
\textit{dor} de experimentar nas suas costas''.

Ele jurou que tinha entendido a minha teoria e acataria os meus
conselhos.

\quebra\section[Os bons cães]{os bons cães\protect\footnote{\uppercase{P}oema escrito na \uppercase{B}élgica, 
onde \uppercase{B}audelaire fizera amizade com \uppercase{J}oseph
\uppercase{S}tevens, pintor de animais --- de cães, principalmente --- e autor do quadro {\itshape \uppercase{O} interior do saltimbanco}, aqui descrito. \uppercase{S}tevens presenteara \uppercase{B}audelaire com um colete seu que o poeta há muito admirava. \uppercase{E}ste escreveu então o
poema dedicado aos cães, em prova de gratidão.}}

\begin{flushright}
\textit{A M. Joseph Stevens} 
\end{flushright}

Nunca me envergonhei, nem perante os jovens escritores do meu século,
da minha admiração por Buffon,\protect\footnote{ Georges Louis Leclerc, 
conde de Buffon, filósofo e naturalista
(1707"-1778), autor de uma volumosa \textit{Histoire Naturelle}.}
mas hoje não é a alma
desse pintor da natureza pomposa que chamarei em meu auxílio. Não.

De muito mais bom grado eu me dirigiria à Sterne, e lhe diria:
“Desça do céu, ou se eleve para mim dos Campos Elíseos, e me inspire em favor dos bons cães, dos pobres cães, um
canto digno de você, sentimental farsista, farsista incomparável! Volte
escarranchado naquele famoso jumento que ainda o acompanha na memória
da posteridade; e que o jumento sobretudo não se esqueça de trazer,
delicadamente pendurado entre os beiços, seu imortal
biscoito''.\protect\footnote{ Referência ao episódio do jumento morto na \textit{Viagem sentimental},
de Laurence Sterne (1713"-1768).}

Vade retro, a musa acadêmica! Não me interessa essa velha melindrosa.
Invoco a musa familiar, urbana, viva, para que me ajude a
cantar os bons cães, os pobres cães, os cães enlameados, esses que
todos evitam como pestíferos e piolhentos, exceto o pobre, de quem são
os sócios, e o poeta, que os mira com olhar fraternal.

Fora, o cão lindinho, esse fátuo quadrúpede, dinamarquês, King
Charles, dogue ou carlindogue, tão encantado consigo mesmo que se
joga indiscretamente nas pernas ou no colo da visita, como que seguro de estar agradando, bagunceiro feito criança, bobo feito uma
rameira, às vezes rabugento e insolente feito um empregado! Fora,
antes de mais nada, essas cobras de quatro patas, trêmulas e desocupadas,
denominadas lebréus, que sequer encerram no focinho pontudo faro
suficiente para seguir a pista de um amigo, e nem, na cabeça
achatada, inteligência para jogar dominó!

Para a toca, esses cansativos parasitas todos!

Que voltem para a sua toca sedosa e acolchoada! Eu canto o cão enlameado,
o cão pobre, o cão sem domicílio, o cão vagabundo, o cão saltimbanco, o
cão cujo instinto, como o do pobre, do cigano e do histrião, é
formidavelmente aguçado pela necessidade, essa mãe tão boa, 
legítima padroeira das inteligências!

Eu canto os cães calamitosos, quer os que vagam, solitários, nos
córregos sinuosos das cidades imensas, quer os que disseram ao
homem abandonado, piscando os olhos maliciosos:
“Leve"-me com você, e com nossas duas misérias talvez
possamos criar alguma espécie de felicidade!''.

\textit{“Para onde vão os cães?''} perguntava antigamente
Nestor Roqueplan\protect\footnote{ Nestor Roqueplan (1804"-1869), 
redator de um folhetim publicado às
segundas"-feiras em \textit{Le Constitutionnel}.}
 num imortal folhetim que ele decerto esqueceu, e que 
só eu, e Sainte"-Beuve\protect\footnote{Charles Augustin 
Sainte"-Beuve (1804"-1869), escritor francês que
se dedicou à crítica e à história literárias.}
talvez, ainda hoje lembramos.

Para onde vão os cães?, perguntam vocês, homens pouco atentos? Vão tratar dos seus negócios.

Encontros de negócios, encontros de amor. Em meio à bruma, em meio à
neve, em meio à lama, sob a canícula mordaz, sob a chuva torrencial,
eles vão, eles vêm, eles trotam, eles passam debaixo dos carros, excitados
pelas pulgas, pela paixão, pela necessidade ou pelo dever. Como

\quebra

\noindent{}nós,
levantaram-se cedo de manhã e estão em busca de sua vida ou correndo para os
seus prazeres.

Há os que dormem numa ruína da periferia e vêm, todo dia, à mesma
hora, pedir esmola na porta de uma cozinha do Palais Royal;
outros acorrem, em bandos, de mais de cinco léguas para
partilhar a refeição preparada pela caridade de algumas donzelas
sexagenárias, cujo coração desocupado se entrega aos bichos porque
os homens, imbecis, já não o querem.

Outros, feito negros fugitivos, enlouquecidos de amor, deixam certos
dias seu distrito para vir à cidade e ficar uma hora saltitando em volta de
uma linda cadela, um pouco negligente em sua toalete, mas
altiva e agradecida.

E são todos muito pontuais, sem agenda ou anotações e sem
carteira.

Vocês conhecem a preguiçosa Bélgica, puderam, como eu, admirar
aqueles cães vigorosos atrelados à carroça do açougueiro, da leiteira ou
do padeiro e demonstram, com seus latidos triunfantes, o orgulhoso prazer
que eles sentem em competir com os cavalos?

Aqui estão dois que pertencem a uma ordem ainda mais civilizada!
Permitam-me introduzi-los no quarto do saltimbanco ausente. Uma
cama de madeira pintada, sem cortinas, cobertores espalhados e
manchados por percevejos, duas cadeiras de palha, uma salamandra, um ou
dois instrumentos musicais avariados. Oh! Que triste mobília! Mas
vejam, por favor, essas duas figuras inteligentes, vestidas com \linebreak

\quebra

\noindent{}roupas
a um tempo puídas e suntuosas, a cabeça coberta qual trovadores ou
militares, vigiando com uma atenção de feiticeiro a \textit{obra sem nome}
que cozinha lentamente na salamandra acesa, e no centro da qual
se ergue uma longa colher, plantada como um destes mastros aéreos a
anunciar que os alicerces da construção estão prontos.

Não será justo que tão zelosos atores não se ponham a caminho sem antes
forrar o estômago com uma sopa forte e consistente? E vocês não
desculpariam um pouco de sensualidade nestes pobres diabos que têm de
enfrentar o dia inteiro a indiferença do público e as injustiças de um
diretor, que leva a parte do leão e come, sozinho, mais sopa que
quatro atores?

Quantas vezes não contemplei, sorridente e comovido, todos esses
filósofos de quatro patas, escravos complacentes, submissos ou
dedicados, que o dicionário republicano poderia afinal qualificar de
\textit{obsequiosos}, se a república, por demais ocupada com a \textit{felicidade} dos
homens, tivesse tempo de considerar a \textit{dignidade} dos cães!

E quantas vezes não pensei que talvez houvesse em algum lugar (quem
sabe, afinal?), para recompensar tanta coragem, tanta paciência e
labor, um paraíso especial para os bons cães, os pobres cães, os cães
enlameados e desolados. Pois se Swedenborg\protect\footnote{ Emmanuel 
Swedenborg (1688-1772), místico sueco muito popular durante o Romantismo (Hedra, 2009).}
afirma que existe um para os turcos e outro para os holandeses!

Os pastores de Virgílio e Teócrito\protect\footnote{ Teócrito, poeta grego (séc. 
\textsc{iii} a.C.) e Virgílio, poeta romano (séc. \textsc{i} a.C.), 
ambos representantes do gênero bucólico e pastoril.} esperavam, em
prêmio por seus cantos alternados, um bom queijo, uma flauta do melhor
artífice ou uma cabra de mamas intumescidas. O poeta que cantou os
pobres cães recebeu em recompensa um lindo colete, de uma cor a um
tempo rica e esmaecida, que lembra os sóis de outono, a beleza das
mulheres maduras e os veranicos de Saint"-Martin.\protect\footnote{ Como são chamados, na França, os
últimos dias quentes que ocorrem às vezes em meados de novembro, quando
o outono já está adiantado.}

Nenhum dos que estavam presentes na taberna da rua Villa"-Hermosa
há de esquecer com que vivacidade o pintor se desfez do seu colete em favor
do poeta, tendo tão bem entendido como é bom e honesto cantar os pobres
cães.

Qual um grandioso tirano italiano dos bons tempos, que ofertava ao divino
Aretino,\protect\footnote{ Aretino (1492"-1557), poeta satírico italiano.}
quer uma adaga enriquecida de pedrarias,
quer um manto de corte, em troca de um precioso soneto ou curioso
poema satírico.

E todas as vezes que o poeta veste o colete do pintor, é obrigado a
pensar nos bons cães, nos cães filósofos, nos veranicos de
Saint"-Martin e na beleza das mulheres muito maduras.

\quebra\section[Epílogo]{epílogo}

\bigskip

\noindent De coração contente, subi até a montanha\\
De onde se contempla a cidade em vastidão,\\
Hospital, lupanares, purgatório, inferno, prisão,\\\medskip

\noindent Onde toda aberração floresce feito flor.\\
Bem sabes, ó Satã, padroeiro da minha aflição,\\
Que não fui até lá para verter um pranto vão;\\\medskip

\noindent Mas, como um velho devasso de uma velha amante,\\
Queria inebriar-me da enorme meretriz\\
Cujo encanto infernal sem cessar me remoça.\\\medskip

\noindent Quer durmas ainda nos lençóis da manhã,\\
Pesada, resfriada, obscura, quer te pavoneies\\ 
Nos véus do entardecer entecidos de ouro fino,\\\medskip

\noindent Eu te amo, ó capital infame! Cortesãs\\
E bandidos, tais os prazeres que amiúde oferecem\\
E não são compreendidos pelo vulgo profano.\\
\vfil

%\pagebreak
%\addcontentsline{toc}{chapter}{Notas}
%\theendnotes

%\def\contentsname{Índice}
% \addcontentsline{toc}{chapter}{Índice}
 %\tableofcontents