\textbf{Charles Baudelaire} (Paris, 1821---\textit{id.}, 1867), escritor francês, é
 ainda hoje reverenciado como um dos paradigmas máximos da criação poética.
 Dono de uma imagética pujante e original, Baudelaire foi também um
 influente crítico de arte e um tradutor de grande envergadura. Alma
 inquieta e conturbada, antípoda da de Goethe, segundo o famoso elogio de
 T. S. Eliot, Baudelaire via com desconfiança a era do progresso,
 entrevendo na modernidade uma morbidez oculta que sua sensibilidade
 extremada não tolerava. Em 1857, a publicação de \textit{As flores do mal}, sua
 obra-prima, ofende a moral burguesa e lhe vale um processo no qual é
 obrigado a pagar uma multa considerável, além de suprimir sete poemas do livro.
 Alguns dos sonetos ali encerrados
 já prefiguravam o simbolismo e o decadentismo, correntes que começavam a
 tomar corpo. Em \textit{Os paraísos artificiais} (1860), explora o potencial
 criador sob o efeito do ópio e do haxixe. Como tradutor, verte
 muitos dos contos e ensaios de  Edgar Allan Poe para o francês, tendo influído assim decisivamente
 para o futuro reconhecimento desse autor, que 
 exerceu influência em sua obra também. Solitário, doente e sem recursos,
 morre em 1867.

\textbf{Pequenos poemas em prosa (O spleen de Paris)}, obra póstuma, publicada em 1869 na reunião
 de escritos do autor feita por Théodore de Banville e Charles Asselineau,
 consumiu mais de dez anos até sua feição definitiva, em 1866. Muitos dos poemas
 já haviam aparecido em jornais e receberam de pronto a estima e a
 admiração da crítica e do público. Figura, em importância, ao lado de
 \textit{As flores do mal} na obra de Baudelaire e ombreia com
 as mais importantes páginas já escritas da literatura universal. Esta edição
saiu à luz pela primeira vez em 1988 pela editora da Universidade de Santa Catarina.

\textbf{Dorothée de Bruchard} é graduada em letras, pela Universidade Federal
 de Santa Catarina, e mestra em Literatura Comparada pela University of
 Nottingham, Inglaterra. Em 1993 funda a Editora Paraula, dedicada à
 publicação de clássicos em edições bilíngues. Traduziu Rousseau, Mallarmé,
 Cendrars, Schwob, e atualmente coordena a \textsc{ong} Escritório do Livro, onde
 pesquisa a história e a arte do livro.

\textbf{Dirceu Villa} é poeta, tradutor e mestre em letras pela Universidade
 de São Paulo. Autor do livro de poemas \textit{Descort} (Hedra, 2003), traduziu
 \textit{Lustra}, de Ezra Pound (inédito) e colabora em diversos veículos de imprensa.


